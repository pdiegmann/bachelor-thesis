\section{What is mHealth?}
\subsection{Definition}
eHealth is defined as "a paradigm involving the concepts of health, technology, and commerce, with commerce and technology as tools in the service of health"\footnote{\cite{MartinezPerez.2013}, p. 2}.
\\
Telehealth stands for the delivery of medical- or health-related information or services via telecommunication technologies.
\\
\\
mHealth in detail was defined as "medical and public health practice supported by mobile devices: such as mobile phones, patient monitoring devices, personal digital assistants (PDAs), and other wireless devices"\footnote{\cite{WorldHealthOrganization.2011} cited by \cite{MartinezPerez.2013}, p. 2}. The introduction of smart phones such as the Apple iPhone and Android devices led to a greater audience and the evolution of mobile tablets further increasing the audience for mHealth purposes. A study relied on the Health Education Curriculum Analysis Tool (HECAT)\footnote{\url{http://www.cdc.gov/HealthyYouth/HECAT/}, last visited on 09/07/2013} to group different mHealth apps together.\footnote{cf. \cite{West.2012}}. HECAT content areas are describing eHealth application content areas, not only content areas for mHealth apps. The study by \cite{West.2012} illustrates the distribution of apps in the HECAT content categories listed in \ref{tab:HECAT}.\footnote{cf. \cite{West.2012}} As \ref{tab:HECAT} illustrates most of the available apps in 2011 in the Apple App Store in the United States of America belonged to the Physical Activity area, whereas drug-related and safety-related apps (like ePill) are the least two categories in terms of magnitude.
\\
From February to May of 2012, a study by \cite{dHeureuse.2012} found several ten thousands of apps in the Google Play Store as well as the Apple App Store limited to the "Health" categories.\footnote{cf. \cite{dHeureuse.2012}, p. 20, Figure 5} The study by \cite{dHeureuse.2012} shows the potential of mHealth for a broader healthcare supported by mobile devices. From March to May of 2012, the total number of apps increased by an average of 6.4\% (Google Play Store) and 4.5\% (Apple App Store) per month.\footnote{cf. \cite{dHeureuse.2012}, p. 20}

\begin{table}[!htb]
    \center
    \begin{tabular}{l | c | c}
        \textbf{HECAT content area} & \textbf{n} & \textbf{\%}\footnotemark \\
        \hline
        Physical Activity & 1108 & 33.21 \\
        \hline
        Personal health and wellness & 962 & 28.84 \\
        \hline
        Healthy eating & 651 & 19.51 \\
        \hline
        Mental and emotional health & 414 & 12.41 \\
        \hline
        Sexual and reproductive health & 243 & 7.28 \\
        \hline
        Alcohol, tobacco, and other drugs & 131 & 3.93 \\
        \hline
        Violence prevention and safety & 96 & 2.88 \\
    \end{tabular}
    \caption[Distribution of Apps related to their HECAT Content Area]{Distribution of Apps related to their HECAT Content Area (N = 3336)\footnotemark}
    \label{tab:HECAT}
\end{table}
\addtocounter{footnote}{-1}
\footnotetext{Apps could be assigned to multiple categories}
\addtocounter{footnote}{1}
\footnotetext{cf. \cite{West.2012}, p. 6, Table 2}

\subsection{mHealth App Categories}
Although \ref{tab:HECAT} lists categories for mHealth apps, it focusses on content and less on the specifics for mHealth apps on other possibly important topics, such as information security or usability. Those content areas range from physical activity to health and wellness (mental, emotional, sexual or diet related) as well as for drugs and prevention. Other literature focusses on data practices and privacy risks with a more technical aspect\footnote{cf. \cite{Njie.2013}, pp. 13-14}.
\\
\\
\cite{Njie.2013} concludes that most of the mHealth apps deal in any way directly or indirectly (e.g. via usage behavior) with sensitive information.\footnote{cf. for this and the following paragraph \cite{Njie.2013}, pp. 13-14, 19, 21} Therefore ten levels of privacy risks were developed and a sample of 43 mHealth and fitness apps were assigned to the different levels. \ref{tab:RiskLevelsmHealth} illustrates the characteristics of every level as well as the distribution of the 43 analyzed apps. 
\\
The risk levels are based on the one hand on the information available to the app and on the other hand on security precautions implemented by the developer to prevent unauthorized access to this information. An important differentiation is also in the anonymity or identifiability of the information accessible by third parties, such as other apps, other developers or unauthorized individuals. The higher the accessibility, the identifiability or the possible harm done by this information, the higher the risk level to be assigned.
\begin{table}[!htb]
    \center
    \begin{tabular}{c | c | p{22.5em} | c}
        \textbf{Level} & \textbf{Risk} & \textbf{Characteristics} & \textbf{\%\footnotemark}\\
        \hline
        9 & Highest & address, financial information, full name, sensitive or embarrassing health (or health-related) information, information that a malicious actor could use to steal or otherwise cause a user to lose money & \multirow{3}[20]{*}{40} \\
        \cline{1-3}
        8 & High & geo-location & \\
        \cline{1-3}
        7 & Medium-high & DOB, ZIP code, any kind of personal medical
information & \\
        \hline
        6 & Medium & risk evaluated to be between level 5 and level 7 & \multirow{3}[12]{*}{32} \\
        \cline{1-3}
        5 & Medium & email, first name, friends, interests, weight, information that is potentially embarrassing or could be used against a person (e.g., in employment)\\
        \cline{1-3}
        4 & Medium & risk evaluated to be between level 5 and level 3\\
        \hline
        3 & Medium-low & anonymized (not personally identifiable) tracking (e.g., app usage), device info, a third party knows the user is using a mobile medical app & \multirow{3}[22]{*}{28} \\
        \cline{1-3}
        2 & Low & risk evaluated to be between level 3 and level 1\\
        \cline{1-3}
        1 & Low & any kind of anonymized data that does not include medical
health-related data or personally identifiable information\\
        \hline
        0 & No & & 0 \\
    \end{tabular}
    \caption[Privacy Risk Levels of mHealth Apps]{Privacy Risk Levels of mHealth Apps (N = 43)\footnotemark}
    \label{tab:RiskLevelsmHealth}
\end{table}
\addtocounter{footnote}{-1}
\footnotetext{of apps in the sample}
\addtocounter{footnote}{1}
\footnotetext{cf. \cite{Njie.2013}, p. 13}
\\
As stated by \cite{Istepanian.2004}, another categorization is possible. They categorized mHealth applications into administrative connectivity, financial connectivity or medical connectivity.\footnote{cf. \cite{Istepanian.2004}, p. 409} Because of the lack of smart phones and a far lesser availability of mobile devices in 2004 compared to today, this article cannot take the recent development in mobile devices into account. The administrative connectivity handles appointments, electronic patient records and any non-financial transactions.\footnote{cf. for this and the two following sentences \cite{Istepanian.2004}, p. 409} The financial connectivity includes all financial transactions such as purchases, billing or any financial services. The third connectivity, the medical connectivity, stands for mobile monitoring and diagnostics.
\\
\\
As described, there are three different categorization approaches for mHealth applications: The content, the information security risk-level and the overall connectivity function. For the content-category as well as the connectivity-category, multiple assignments are possible. Combined these categorization approaches form a specific grouping of mHealth apps. Depending on the categorization in the privacy risk, one can take precautionary measures. With the categorization in a HECAT content area one can identify the target audience more precisely as well as with the help of the connectivity category.

\subsection{Classification of the ePill Web Application}
ePill can be categorized in the HECAT content areas as "Alcohol, tobacco, and other drugs", because it informs about (medical) drugs. Since ePill also informs about adverse effects and interactions between pharmaceuticals, it belongs furthermore to the content area "Violence prevention and safety".
\\
\\
The ePill web application is not connected to any electronic patient records, nor does it store any user related information such as a history of the last searched pharmaceuticals, but it does not utilize SSL-encryption. Therefore it might not be collecting information or storing anything, but third parties could collect user specific information by monitoring.
\\
Putting this information into context with the risk levels developed by \cite{Njie.2013}, if SSL-encryption would be utilized by the ePill web application, it could be categorized as level three. With SSL-encryption, third parties could still retrieve browser and OS specific information, but not the data sent and retrieved with each request such as pharmaceutical information. Without encryption all data sent and retrieved is visible to possible eavesdropper. With information about searched pharmaceuticals, one could assemble an overall picture of the ingested medications and therefore extrapolate possible diseases. Still, all data is anonymized.
\\
Having in mind, that ePill still is in early prototyping and assuming, that the SSL-encryption will be an upcoming feature, the risk is more of a medium to low level. Dealing with anonymous data only and protecting them with encryption leaves very little room for serious risks. We would therefore categorize ePill as a level two in terms of privacy risk levels.
\\
\\
Although ePill does not absolutely fit in any of the connectivity categories, its closest fit is within medical connectivity. Because of the aim to provide pharmaceutical (therefore medical) information, it could still be found within the medical connectivity category.
\\
\\
Concluding this categorization, we would suggest to categorize the ePill web application as a low privacy risk, drug and safety-related medical connectivity mHealth application.

\subsection{Why is a special Focus on mHealth Apps warranted?}
mHealth apps differ in some way from general (mobile) applications but also from eHealth applications. While mHealth apps can be used in many different situations and with very different intentions, the special focus on e.g. equality of all users and accessibility for all possible users is important for mHealth apps. This is supported by the definition of mHealth apps: "mHealth apps aim at providing seamless, global access to tailored health IT services and have the potential to alleviate global health burdens"\footnote{\cite{Dehling.2013}, p. 1}, which means that they should be accessible by mostly all possible users. Whereas other types of apps do not necessarily need to be accessible by any user, we want to stress that accessibility does not only mean usability (especially for elderly people), but also global for different social layers or cultures\footnote{cf. \cite{Dehling.2013}, p. 1}.
\\
\\
mHealth apps deal with medical- or health-related information and have therefore to deal with sensitive information and have to address privacy risks and concerns. As pointed out by \cite{Njie.2013} and already referred to in \ref{tab:RiskLevelsmHealth}, many mHealth apps deal with highly sensitive data and have serious privacy risks. \cite{Dehling.2013} illustrates the possible damages through leaks, manipulation or loss of information, such as loss of affection and reputation or lessened employment possibilities.\footnote{cf. \cite{Dehling.2013}, p. 7}