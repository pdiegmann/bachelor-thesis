\section{What is mHealth?}

\subsection{Definition}
mHealth, also known as m-Health, is an abbreviation for mobile health and is a refinement of eHealth (or e-Health, an abbreviation for electronic health), which itself belongs to the field of telehealth.\footnote{cf. \cite{MartinezPerez.2013}, p. 2}
\\
eHealth is defined as "a paradigm involving the concepts of health, technology, and commerce, with commerce and technology as tools in the service of health".\footnote{\cite{MartinezPerez.2013}, p. 2}
\\
Telehealth means the delivery of medical- or health-related information or services via telecommunication technologies.
\\
mHealth in detail is defined as "medical and public health practice supported by mobile devices, such as mobile phones, patient monitoring devices, personal digital assistants (PDAs), and other wireless devices".\footnote{\cite{WorldHealthOrganization.2011} cited by \cite{MartinezPerez.2013}, p. 2} The introduction of smart phones like the Apple iPhone or any Android device led to a greater audience and the evolution of mobile tablets further increased the audience for mHealth purposes. Some studies rely on the Health Education Curriculum Analysis Tool (HECAT)\footnote{\url{http://www.cdc.gov/HealthyYouth/HECAT/}} to group different mHealth apps together. A study\footnote{cf. \cite{West.2012}} illustrates the distribution of apps in different categories. As ~\ref{tab:HECAT} illustrates, most of the available apps in 2011 in the Apple App Store in the United States of America belonged to the Physical Activity area, whereas drug-related and safety-related apps (like ePill) are the least two. 

\begin{table}[!htb]
    \center
    \begin{tabular}{l | c | c}
        \textbf{HECAT content area} & \textbf{n} & \textbf{\%}\footnotemark \\
        \hline
        Physical Activity & 1108 & 33.21 \\
        \hline
        Personal health and wellness & 962 & 28.84 \\
        \hline
        Healthy eating & 651 & 19.51 \\
        \hline
        Mental and emotional health & 414 & 12.41 \\
        \hline
        Sexual and reproductive health & 243 & 7.28 \\
        \hline
        Alcohol, tobacco, and other drugs & 131 & 3.93 \\
        \hline
        Violence prevention and safety & 96 & 2.88 \\
    \end{tabular}
    \caption[HECAT Content Area App Distribution]{HECAT Content Area App Distribution (N = 3336)\footnotemark}
    \label{tab:HECAT}
\end{table}
\addtocounter{footnote}{-1}
\footnotetext{Apps could be added to multiple categories}
\addtocounter{footnote}{1}
\footnotetext{cf. \cite{West.2012}, p. 5, Table 2}

\subsection{mHealth App Categories}
Although the ~\ref{tab:HECAT} listed categories for mHealth apps, it focusses on content or functionality and less on the specifics for mHealth apps on other topics, such as information security or usability. Other literature focusses on data practices and privacy risks with a more technical aspect\footnote{cf. \cite{Njie.2013}, pp. 13-14} 
or into administrative connectivity, financial connectivity or medical connectivity.\footnote{cf. \cite{Istepanian.2004}, p. 6} The third categorization was stated already in 2004, so this article cannot take the recent development in mobile devices into account. Nevertheless the categorization is still appropriate. The administrative connectivity handles appointments, electronic patient records and any non-financial transactions, the financial connectivity handles all financial transactions like purchases, billing or any financial services. The third connectivity, the medical connectivity, handles mobile monitoring and diagnostics.

\subsection{Classification of the ePill Web Application}

\subsection{Why is a special Focus on mHealth Apps warranted?}