\section{What is mHealth?}

\subsection{Definition}
\todo{More Detail!}
\\
\\
mHealth, also known as m-Health, is an abbreviation for mobile health and is a refinement of eHealth (or e-Health, an abbreviation for electronic health), which itself belongs to the field of telehealth.\footnote{cf. \cite{MartinezPerez.2013}, p. 2}
\\
\\
eHealth is defined as "a paradigm involving the concepts of health, technology, and commerce, with commerce and technology as tools in the service of health".\footnote{\cite{MartinezPerez.2013}, p. 2}
\\
\\
Telehealth means the delivery of medical- or health-related information or services via telecommunication technologies.
\\
\\
mHealth in detail is defined as "medical and public health practice supported by mobile devices, such as mobile phones, patient monitoring devices, personal digital assistants (PDAs), and other wireless devices".\footnote{\cite{WorldHealthOrganization.2011} cited by \cite{MartinezPerez.2013}, p. 2} The introduction of smart phones like the Apple iPhone or any Android device led to a greater audience and the evolution of mobile tablets further increased the audience for mHealth purposes. A study\footnote{cf. for this and the first following sentence \cite{West.2012}} relied on the Health Education Curriculum Analysis Tool (HECAT)\footnote{\url{http://www.cdc.gov/HealthyYouth/HECAT/}} to group different mHealth apps together. This study illustrates the distribution of apps in different categories. As \ref{tab:HECAT} illustrates, most of the available apps in 2011 in the Apple App Store in the United States of America belonged to the Physical Activity area, whereas drug-related and safety-related apps (like ePill) are the least two. 
\begin{table}[!htb]
    \center
    \begin{tabular}{l | c | c}
        \textbf{HECAT content area} & \textbf{n} & \textbf{\%}\footnotemark \\
        \hline
        Physical Activity & 1108 & 33.21 \\
        \hline
        Personal health and wellness & 962 & 28.84 \\
        \hline
        Healthy eating & 651 & 19.51 \\
        \hline
        Mental and emotional health & 414 & 12.41 \\
        \hline
        Sexual and reproductive health & 243 & 7.28 \\
        \hline
        Alcohol, tobacco, and other drugs & 131 & 3.93 \\
        \hline
        Violence prevention and safety & 96 & 2.88 \\
    \end{tabular}
    \caption[HECAT Content Area App Distribution]{HECAT Content Area App Distribution (N = 3336)\footnotemark}
    \label{tab:HECAT}
\end{table}
\addtocounter{footnote}{-1}
\footnotetext{Apps could be added to multiple categories}
\addtocounter{footnote}{1}
\footnotetext{cf. \cite{West.2012}, p. 5, Table 2}
\\
From February to May of 2012, a Study by \cite{dHeureuse.2012} found several ten thousands of apps in the Google Play Store as well as the Apple App Store just in the "Health" categories.\footnote{cf. \cite{dHeureuse.2012}, p. 20, Figure 5} This study shows the potential of mHealth for a broader healthcare supported by mobile devices. From March to May of 2012, the total number of apps increased by an average of 6.4\% (Google Play Store) and 4.5\% (Apple App Store) per month.\footnote{cf. \cite{dHeureuse.2012}, p. 20}

\subsection{mHealth App Categories}
Although the \ref{tab:HECAT} listed categories for mHealth apps, it focusses on content and less on the specifics for mHealth apps on other possibly important topics, such as information security or usability. Other literature focusses on data practices and privacy risks with a more technical aspect\footnote{cf. for this and the following three sentences \cite{Njie.2013}, pp. 13-14}. \cite{Njie.2013} concludes that most of the mHealth apps deal in any way with directly or indirectly (e.g. via usage behavior) with sensitive information. Therefor ten levels of privacy risks were developed and a sample of 43 mHealth and fitness apps were assigned to the different levels. \ref{tab:RiskLevelsmHealth} illustrates the characteristics of every level as well as the distribution of the 43 analyzed apps.
\\
\\
As stated by \cite{Istepanian.2004}, another categorization is possible. They categorized mHealth applications into administrative connectivity, financial connectivity or medical connectivity.\footnote{cf. \cite{Istepanian.2004}, p. 6} Because of the lack of smart phones and a far lesser availability of mobile devices in 2004 compared to today, this article cannot take the recent development in mobile devices into account. Nevertheless the categorization is still appropriate. The administrative connectivity handles appointments, electronic patient records and any non-financial transactions, the financial connectivity handles all financial transactions like purchases, billing or any financial services.\footnote{cf. for this and the first following sentence \cite{Istepanian.2004}, p. 13} The third connectivity, the medical connectivity, handles mobile monitoring and diagnostics.
\\
\begin{table}[!htb]
    \center
    \begin{tabular}{c | c | p{22.5em} | c}
        \textbf{Level} & \textbf{Risk} & \textbf{Characteristics} & \textbf{\%}\\
        \hline
        9 & Highest & address, financial information, full name, sensitive or embarrassing health (or health-related) information, information that a malicious actor could use to steal or otherwise cause a user to lose money & \multirow{3}[20]{*}{40} \\
        \cline{1-3}
        8 & High & geo-location & \\
        \cline{1-3}
        7 & Medium-high & DOB, ZIP code, any kind of personal medical
information & \\
        \hline
        6 & Medium & risk evaluated to be between level 5 and level 7 & \multirow{3}[12]{*}{32} \\
        \cline{1-3}
        5 & Medium & email, first name, friends, interests, weight, information that is potentially embarrassing or could be used against a person (e.g., in employment)\\
        \cline{1-3}
        4 & Medium & risk evaluated to be between level 5 and level 3\\
        \hline
        3 & Medium-low & anonymized (not personally identifiable) tracking (e.g., app usage), device info, a third party knows the user is using a mobile medical app & \multirow{3}[22]{*}{28} \\
        \cline{1-3}
        2 & Low & risk evaluated to be between level 3 and level 1\\
        \cline{1-3}
        1 & Low & any kind of anonymized data that does not include medical
health-related data or personally identifiable information\\
        \hline
        0 & No & & 0 \\
    \end{tabular}
    \caption[Privacy Risk Levels of mHealth Apps]{Privacy Risk Levels of mHealth Apps (N = 43)\footnotemark}
    \label{tab:RiskLevelsmHealth}
\end{table}
\footnotetext{cf. \cite{Njie.2013}, p. 13}
\\
There are there different sub-categories for mHealth applications: The content, the information security risk-level and the overall connectivity function. For the content-category as well as the connectivity-category, multiple assignments are possible. Combined these sub-categories form a specific grouping of mHealth apps. Depending on the categorization in the privacy risk, one can easily take care for precautions. With the categorization into a HECAT content area, one can identify the target audience more precisely as well as with the help of the connectivity category.

\subsection{Classification of the ePill Web Application}
ePill is to be categorized in the beforehand mentioned HECAT content areas mainly as "Alcohol, tobacco, and other drugs", because of the purpose to inform about (medical) drugs. Additionally, ePill informs about adverse effects and interactions, so it also belongs to the content area of "Violence prevention and safety".
\\
\\
The ePill web application is not connected to any electronic patient records, nor does it store any user related information, like the last searched pharmaceuticals. But it does not utilize SSL-encryption. Therefor it might not be collecting information or storing anything, but third parties could collect user specific information by monitoring.
\\
Setting this information into context with the risk levels developed by \cite{Njie.2013}, the ePill web application could be categorized as level three, if SSL-encryption would be utilized. If that would be the case, third parties could retrieve browser and OS specific information, but not data sent and retrieved with each request like pharmaceutical information. Without encryption, all data sent and retrieved is visible to possible eavesdropper. With information about searched pharmaceuticals, one could assemble a overall picture of the ingested drugs and therefor extrapolate possible diseases. Still, all data is anonymized.
\\
Having in mind, that ePill still is in early prototyping and assuming, that the SSL-encryption will follow, the risk is more of a medium to low level. Dealing with only anonymous data and protecting them with encryption leaves only very less room for serious risks. We would therefor categorize ePill in terms of privacy risk levels as a level two.
\\
\\
Although ePill does not fit absolutely in any of the connectivity categories, it fits best into the medical connectivity. Because of the aim to provide pharmaceutical (therefor medical) information, it belongs definitely to the medical connectivity category.
\\
\\
Concluding this categorization, we would suggest to categorize the ePill web application as a low privacy risk, drug- and safety-related medical connectivity mHealth application.
\\
The ePill web application lacks a optimization for mobile devices but all categorizations match their definition. The HECAT content area is by definition not limited to mobile devices and privacy risks are in many ways the same for mobile apps and web applications.

\subsection{Why is a special Focus on mHealth Apps warranted?}
mHealth apps differ in some way from general (mobile) applications but also from eHealth applications. While mHealth apps can be used in many different situations and with very different intentions, the special focus on e.g. equality of all users and accessibility for all possible users are not as important for other areas of mobile apps as they are for mHealth apps. 
\\
\\
mHealth apps are defined to "aim at providing seamless, global access to tailored health IT services and have the potential to alleviate global health burdens."\footnote{\cite{Dehling.2013}, p. 1}, which means, that they should be accessible by mostly all possible users, whereas other types of apps do not necessarily need to be accessible by any user. We want to stress, that accessibility does not only mean usability (especially for elderly people), but also e.g. different social layers or cultures.
\\
\\
Furthermore, mHealth apps deal with medical- or health-related information and have therefor to deal wit sensitive information and are to address privacy risks and concerns. As pointed out by \cite{Njie.2013} and already referred to in \ref{tab:RiskLevelsmHealth}, many mHealth apps deal with highly sensitive data and have serious privacy risks. \cite{Dehling.2013} illustrate the possible damages through leaks, manipulation or loss of information.\footnote{cf. \cite{Dehling.2013}, p. 7}
\\
\\
To address these concerns and issues in a mHealth project, they need to be made clear and experiences must be shared as well as interpreted. The following chapter will present all experiences made during the development of a mobile frontend for ePill in a structured way. We will list all theoretical preconditions, outline the analysis as well as the implementation of the mHealth app. Afterwards we will validate the product and give an overview about the lessons we learned.