\section{The ePill System}
\subsection{The System in General}
The ePill system (http://epill.uni-koeln.de) was developed by the University of Cologne to improve the readability and comprehensibility of instruction leaflets contained within the packaging of pharmaceutical drugs. Additionally ePill aims to provide further information on adverse reactions and interactions of different medical drugs. ePill emphasizes an easy readability and access to informations.
\\
ePill is currently a prototype of the planned system, used only for research purposes and it is only actively used by the University of Cologne. Therefore it is only localized in German and contains just pharmaceuticals available in Germany. ePill utilizes the "GELBE LISTE PHARMINDEX"\footnote{\url{http://www.gelbe-liste.de}}, provided by Medizinische Medien Informations GmbH MMI as main source for pharmaceutical information.
\\
\\
The system provides three major functions: searching for pharmaceuticals, display information on pharmaceuticals and supplementing services.\footnote{cf. for this paragraph \cite{Dehling.2012}, p. 2 and \cite{Dehling.2012b}, p. 5} The search enables the user to find corresponding pharmaceuticals depending on specified parameters in the underlying database. As an extend, the display functionality enables the user to read the leaflet information in an optimized fashion. Finally supplementing services are provided to refine the displayed information (e.g. select the level of detail of the displayed information). Comparing pharmaceuticals side by side as well as a list of adverse reactions or side effects can be displayed for a list of pharmaceuticals.
\\
\\
An integration and personalization depending on the current user's health records was not implemented due to the arising privacy and trust challenges.\footnote{cf. \cite{Kaletsch.2011} cited by \cite{Dehling.2012}, p. 2}\footnote{cf. \cite{Kaletsch.2011}, pp. 5-6} Instead the system was focused on patient friendliness regarding readability and comprehensibility.\footnote{cf. \cite{Dehling.2012b}, p. 2}
\\
\\
The system uses a Model-View-Controller (MVC) architecture\footnote{cf. \cite{Dehling.2012}, p. 3} and utilizes a relational database as persistent data storage.\footnote{cf. \cite{Dehling.2012}, p. 5 for this and the following two sentences} The data is organized with the pharmaceutical ingredients in mind. Products are any pharmaceuticals, which may contain specific molecules, which themselves may be related to specific adverse reactions with other molecules. 
\\
With this atomized organization of the pharmaceutical information, it becomes more easier to compare different pharmaceuticals and have very consistent information about molecules and adverse reactions with other pharmaceuticals.

\subsection{The Web Application}
The web application of the ePill system introduces itself highly customizable to the user. It offers the user to choose between a default view, a customizable view and an expert view. The default view aims to provide all necessary information in a compact way. The customizable view offers more choices for the controls and user interface elements to be displayed. The expert view activates the most detailed information level. The pharmaceutical informations to be displayed can be fine tuned for every view. ePill offers four different presets varying from only the most basic up to all available information. These presets can be further customized by afterwards selecting or deselecting items. ePill offers customization for the displayed leaflet information as well as the visibility of controls for access to supplementing services and settings in the sidebar. Additionally the font-size can be set to normal, bigger and biggest to support visually impaired users.
\\
\\
Three columns shape the layout. The first, leftmost column contains the main navigation for searching, pharmaceutical listings, basic functionality such as help pages and settings as well as extended functionality such as interactions research and adverse reaction lookup or pharmaceutical comparisons. The second, centered column contains the current content. This column has tabs\footnote{user interface elements which imitate the look and functionality of card tabs inserted in paper files}, which can be assigned different contents. With this tabular layout, e.g. multiple, different search queries can easily be switched and held in parallel. The third, rightmost column can be used to dynamically display or hide selected information. Depending on the beforehand selected view, the left or right columns are hidden or visible. On the pharmaceutical detail page, the website offers the user the functionality to explain any term as well as a shortcut to the page's top.
\\
\\
The specific content layout is very consistent. Headlines are made salient and the arrangement of common sections are congruent. Changes in settings are applied with no delay and without a page reload.
\\
\\
Although this web application is not optimized for mobile applications and designed with a desktop computer in mind, it can be accessed by nearly any modern mobile computing device, such as a smart phone or a tablet, and can therefore categorized as a mHealth application. This categorization is important to the following section, to clarify the differences between the web application and the mobile client.