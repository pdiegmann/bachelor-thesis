\section{The ePill System}

\subsection{The System in general}
The ePill system (http://epill.uni-koeln.de) was developed by the University of Cologne to improve the readability and comprehensibility of instruction leaflets of medical drugs. Additionally ePill aims to provide further information on adverse reactions and interactions of different medical drugs. ePill emphasizes an easy readability and access to informations.
\\
There are three major functions covered by the system: Searching for pharmaceuticals, display information on pharmaceuticals and supplementing services.\footnote{cf. for this section \cite{Dehling.2012}, p. 2} The search enables the user to find corresponding pharmaceuticals depending on specified parameters in the underlying database. As an extend, the display functionality enables the user to read the leaflet information in an optimized fashion. Finally supplementing services are provided to refine the displayed information (e.g. select the level of detail of the displayed information), linking pharmaceuticals as well as other information and aggregate pharmaceutical information (e.g. interactions).
\\
An integration and personalization depending on the current user's health records was not implemented due to the arising privacy and trust challenges.\footnote{cf. \cite{Kaletsch.2011} cited by \cite{Dehling.2012}, p. 2}\footnote{cf. \cite{Kaletsch.2011}, pp. 5-6}

\subsection{The Web Application}
The web application of the ePill system introduces itself highly customizable to the user. Right 