\section{The ePill System}
\subsection{The System in General}
The ePill system (http://epill.uni-koeln.de) was developed by the University of Cologne to improve the readability and comprehensibility of instruction leaflets contained within the packaging of pharmaceutical drugs. Additionally ePill aims to provide further information on adverse reactions and interactions of different medical drugs. ePill emphasizes an easy readability and access to informations.
\\
ePill is currently a prototype of a system, used only for research purposes and it is only actively used by the University of Cologne. Therefor it is only localized in German and contains only pharmaceuticals available in Germany. ePill utilizes the "GELBE LISTE PHARMINDEX"\footnote{\url{http://www.gelbe-liste.de}}, provided by Medizinische Medien Informations GmbH MMI.
\\
\\
There are three major functions covered by the system: searching for pharmaceuticals, display information on pharmaceuticals and supplementing services.\footnote{cf. for this paragraph \cite{Dehling.2012}, p. 2 and \cite{Dehling.2012b}, p. 5} The search enables the user to find corresponding pharmaceuticals depending on specified parameters in the underlying database. As an extend, the display functionality enables the user to read the leaflet information in an optimized fashion. Finally supplementing services are provided to refine the displayed information (e.g. select the level of detail of the displayed information), linking pharmaceuticals as well as other information and aggregate pharmaceutical information (e.g. interactions). 
\\
\\
An integration and personalization depending on the current user's health records was not implemented due to the arising privacy and trust challenges.\footnote{cf. \cite{Kaletsch.2011} cited by \cite{Dehling.2012}, p. 2}\footnote{cf. \cite{Kaletsch.2011}, pp. 5-6}
\\
\\
The system uses a Model-View-Controller (MVC) architecture\footnote{cf. \cite{Dehling.2012}, p. 3} and utilizes a relational database as persistent data storage.\footnote{cf. \cite{Dehling.2012}, p. 5 for this and the following two sentences} The data is organized in an atomic way. Products are any pharmaceuticals, which may contain specific molecules, which themselves may be related to specific adverse reactions with other molecules. 
\\
With this atomized organization of the pharmaceutical information, it becomes more easily to compare different pharmaceuticals and have very consistent information about molecules and adverse reactions for different pharmaceuticals.

\subsection{The Web Application}
The web application of the ePill system introduces itself highly customizable to the user. It offers the user the choice between a default view, a customizable view and an expert view. The default view aims to provide all necessary information in a compact way. The customizable view offers more choices for the elements to be displayed. The expert view activates all options for the most detailed information level. The pharmaceutical informations to be displayed can be fine tuned for every view. ePill offers four different presets varying from only the most basic up to all available information. These presets can be further customized by afterwards selecting or deselecting items. Additionally the font-size can be set to normal, bigger and biggest to support visually impaired users.
\\
\\
\todo{Kristy's remarks}
Three columns shape the layout. The leftmost column contains the main navigation for searching, pharmaceutical listings, basic functionality like help pages and settings as well as extended functionality like interactions research and adverse reaction lookup or pharmaceutical comparisons. The centered column contains the current content. This column has tabs, which can be assigned different contents. With this tabular layout, e.g. multiple, different search queries can easily be switched and held in parallel. The rightmost column can be used to dynamically display or hide specific information. Depending on the beforehand selected view, the left or right columns are hidden or visible. The website also offers the user on the pharmaceutical detail page to explain any term as well as a shortcut to the page's top.
\\
\\
The specific content layout is very consistent. Headlines are made salient and the arrangement of common sections are congruent. Changes in settings are applied with no delay and without a page reload. Any changes are made congruent with the chosen layout and other related settings.
\\
\\
Although this web application is not optimized for mobile applications and designed with a desktop computer in mind, it can be accessed by nearly any modern mobile computing device, like a smart phone or a tablet, and can therefor categorized as a mHealth application. This association is important to the following section, to clarify the differences between this web application and the mobile client, because with this assumption we can categorize on the same level and focus on the essential differences.