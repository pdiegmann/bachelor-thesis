\section{Conclusion}
\todo{Fuck this shit}
\\
\\
This thesis aimed to state different best practices and guidelines and explain, based on a mobile prototype, how those can be implemented.
\\
\\
We collected existing best practices, guidelines and norms for mobile apps as well as specific ones for mHealth apps. For a discussion we performed a short literature review as stated in \ref{subsec:BestPractices} \nameref{subsec:BestPractices}. All found were discussed and their importance was rated for the ePill project. This review showed that not many guidelines or theoretical development frameworks specialized on mHealth apps or even more general mobile apps are available and that some guidelines are in need for revision.
\\
\\
We explained the different phases we passed during the development and stated our experiences. Some of the different available frameworks were introduced and discussed. We gave reasons for our choice for Vaadin and TouchKit.
\\
\\
The app developed during this thesis corresponds to the requirements developed in sections \ref{subsec:Preconditions} \nameref{subsec:Preconditions}, \ref{subsec:Analysis} \nameref{subsec:Analysis} and \ref{subsec:Planning} \nameref{subsec:Planning}. All modifications or deviations were discussed and explained. The developed app contains all needed functionality and has a mobile optimized user interface which is available on nearly any modern mobile device.
\\
\\
A possible point of criticism is that we did not have the time for concentrating on fine tuning and deep testing of the mobile app. Especially the missing knowledge of Vaadin and TouchKit resulted in a user interface planning which was not accurate enough. We had to find workarounds for the mentioned missing controls we integrated in our planning. This was a time consuming process we could have had avoided with more knowledge. Those workarounds could be further improved.
\\
\\
Adding and rating the subgoals, we come to the conclusion that this thesis' goal is achieved and that we gave not only an overview about the current state of guidelines for mHealth apps, but also developed a functional prototype of a mobile frontend for the ePill web application.