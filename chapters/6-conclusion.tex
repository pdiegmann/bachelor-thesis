\section{Conclusion}
Adding and rating the subgoals, we come to the conclusion that this thesis' goals are achieved and that we gave not only an overview about the current state of guidelines for mHealth apps, but also developed a functional prototype of a mobile frontend for the ePill web application. We stated all processes we passed through, as we aimed to do. 
\\
\\
The literature review showed that not many guidelines or theoretical development frameworks specialized on mHealth apps or even for mobile apps in general are available. Some guidelines are in need for revision, due to the changes in mobile development since the release of the literature. The goal to provide a short overview over the current state of best practices and guidelines is thereby fulfilled.
\\
\\
We reported any issues and challenges we discovered during the development of a mobile frontend for the ePill system in sections \ref{subsec:Planning} and \ref{subsec:Implementation} and summarized them in section \ref{sec:LessonsLearned}. The goal for providing an overview over the lessons learned during the systematic development is thereby fulfilled. 
\\
A validation of the developed mobile frontend as required by the objectives of this thesis was performed in section \ref{subsec:Validation}. The validation confirmed the development of the mobile frontend in compliance with the stated guidelines and best practices. The goal for developing a mobile frontend and its validation is thereby fulfilled.
\\
\\
All modifications or deviations between the web application and the mobile app or between the planning and the implementation were discussed and explained, such as the altered search bar in sections \ref{subsec:Implementation} and \ref{subsec:Validation}. The developed mobile app contains all needed functionality and has a mobile optimized user interface which is available on nearly any modern mobile device. It is therefore in compliance with the aim to provide a cross platform mobile frontend for ePill.
\\
\\
A possible point of criticism is that there was no time for concentrating on fine tuning and deep testing of the mobile app. The missing knowledge of Vaadin and TouchKit resulted in a user interface planning which was not accurate. We had to find workarounds for the missing user interface controls we integrated in our planning. Additional knowledge about Vaadin and TouchKit would have avoided this time consuming process. Those workarounds could be further improved by refining the CSS styling or implement additional user interface controls for Vaadin and TouchKit.
\\
\\
Additional literature for designing user interfaces in general could have been reviewed. For example "The Design of Everyday Things" by \cite{Norman2002} could have been integrated and compared to the "Three Layers Design Guideline" for mobile application by \cite{AyobNurulZakiahbinti.2009}. These additional general design guidelines could have improved the user interface design.
\\
\\
Another point of criticism is that some of the discovered issues are specific to Vaadin and TouchKit, such as the missing user interface controls or the missing influence on the final user interface appearance. Other issues, such as the missing familiar look of the mobile app on Android or Windows Phone devices are applicable to any app which user interfaces are not defined independently for every OS.