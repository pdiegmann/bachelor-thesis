% SCHLUSSBETRACHTUNG (1-2 Seiten)
\section{Schlussbetrachtung} 
\label{sec:Schluss}

% Wichtige Ergebnisse/Kernthesen, nicht Zusammenfassung der Arbeit
In dieser Bachelorarbeit wurde ein wissenschaftlicher Literaturreview für den Zeitraum Mai 2012 bis April 2013 durchgeführt, um Lösungsvorschläge zu bestehenden Herausforderungen im \TCC vorzustellen.
Dazu wurden die in Kapitel \ref{sec:Systematik} genannten Online-Portale mit vordefinierten Suchstrings durchsucht und die Ergebnisse ausgewertet.
\newline
% Kritische Relexion der Ergebnisse im Hinblick auf Problemstellung
% - Aufzeigen der Grenzen der produzierten Ergebnisse
Aufgrund des limitierten Umfangs einer Bachelorarbeit mussten beim Inhalt dieser Arbeit einige Begrenzungen vorgenommen werden. 
Zunächst konnte im Rahmen dieser Arbeit nur auf die wichtigsten zwei Stakeholder des Cloud Computings eingegangen werden. Neben \CSUn und \CSPn hätte man zusätzlich die von \cite{Marston.2011} erwähnten \emph{Ermöglicher} von \CC Projekten und die \emph{Regulatoren} von \CSs erörtern können.
\newline
Auch der Umfang der im Rahmen der Arbeit durchgeführten Literaturrecherche ist eingeschränkt.
Wie in Abschnitt \ref{sec:Systematik} beschrieben, wurden die gefundenen Konferenzbeiträge aufgrund der großen Ergebnismenge nicht einzeln, anhand der vordefinierten \HiTCC betrachtet, sondern lediglich eine Volltextsuche in den Abstracts durchgeführt. 
Da die Beschreibung der Artikel wissenschaftlicher Fachzeitschriften bereits den Umfang dieser Bachelorarbeit komplett ausfüllt, wurde die Volltextsuche in den Abstracts der Konferenzbeiträgen auf die Herausforderungen der Integrationsphase und der Außerdienststellungsphase beschränkt.
Diese beiden Phasen wurden, wie in Tabelle 3-2 zu sehen, in der Fachliteratur wenig behandelt. 
Die Suche nach den speziellen Herausforderungen dieser Lebenszyklus-Phasen  brachte aber auch in den restlichen Konferenzbeiträgen keine weiteren Ergebnisse.
\newline
% Beurteilung der Zielerreichung/Kritische Würdigung der eigenen Arbeit
% - eigene Vorgehensweise
In Abschnitt \ref{sec:Herausforderungen} wurden die in der Literatur gefundenen Herausforderungen entsprechend ihrer Themengebiete gruppiert und erläutert.
Dadurch wurde das erste Teilziel der Bachelorarbeit, die Identifizierung der behandelten Herausforderungen, erreicht.
Nachdem die Herausforderungen des Themengebiets \CC erarbeitet wurden, beschreibt Abschnitt \ref{sec:QuantAnalysis} die verwendete Systematik, mit der der Literaturreview durchgeführt wurde und gibt durch eine quantitative Analyse der gefundenen Lösungsvorschläge eine Übersicht, über die Bandbreite der gefundenen Lösungen.
Das dritte Teilziel, die Recherche nach den konkreten Lösungsvorschlägen, wurde in Abschnitt \ref{sec:Loesungsvorschlaege} durch eine detailliertere Beschreibung der relevanten Artikel, erfüllt.
\newline
Im Laufe dieser detaillierteren Betrachtung fiel, wie bereits in Abschnitt \ref{sec:QuantAnalysis}, auf, dass einige Herausforderungen aus dem \TCC durch die wissenschaftliche Fachliteratur und in den Konferenzbeiträgen nur sehr wenig oder gar nicht adressiert wurden. 
Aus diesem Grund schließt Abschnitt \ref{sec:Zukunft}, mit dem vierten Teilziel, einem Ausblick auf zukünftige Forschung, ab.
Nimmt man diese vier Teilziele zusammen, ergeben sie eine umfassende Ausarbeitung die das Hauptziel dieser Bachelorarbeit erreicht. 
Die Frage, wie \HiTCC in wissenschaftlicher Fachliteratur und Konferenzbeiträgen von Mai 2012 bis April 2013 adressiert wurden konnte somit beantwortet werden. 
Somit spiegelt die Arbeit den State-of-the-Art des Cloud Computing wieder.
\newline
% Ggf. potentielle Kritikpunkte an der eigenen Arbeit antizipieren und Stellung beziehen
Ein möglicher Kritikpunkt an der Vorgehensweise in dieser Arbeit ist, dass nicht jeder gefundene Lösungsansatz ausführlich genug beschrieben werden konnte, wie es der jeweils präsentierte Ansatz benötigen würde. Leider ist im Rahmen dieser Bachelorarbeit eine detailreichere Darstellung der Lösungsansätze nicht möglich und es konnten lediglich die Grundideen kommuniziert werden.
Dennoch wurde versucht die vorgestellten Lösungsvorschläge so umfassend und treffend wie möglich zu beschreiben, sodass der Leser einen Eindruck über die Kerngedanken der beschriebenen Artikel erhalten konnte.