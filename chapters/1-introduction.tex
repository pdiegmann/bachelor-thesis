\section{Introduction}
\subsection{Research Problem}
While it has become easy to develop a mobile health (mHealth) application (app), there is much more to it than just the aspects of the app's core functionality. Currently only very few guidelines, best practices and systematic development approaches for mobile app development can be found. Furthermore even less can be found for the specific area of mHealth apps.
\\
\\
Security leaks or even abuse of private and sensitive information\footnote{information, which is personal. Can be related to financial-, health- or otherwise personal relevant information, suggested by \cite{FutureofPrivacyForumCenterforDemocracy&Technology.2011}, p. 6, although the definition varies} can lead to great harm for the app user and to legal issues for the developer. Abuse of personal health related information can result in loss of reputation (e.g. sexual transmitted diseases) or financial drawbacks and decreased chances of employment (e.g. chronic diseases, genetic dispositions)\footnote{cf. \cite{Dehling.2013}, pp. 6-7}. With poorly developed apps, there is a danger of security leaks and hence for data abuse. Thus the risk for app users increases. A study has shown that very few mHealth apps entail little or low risk for the app user.\footnote{cf. \cite{Njie.2013}, pp. 19-20} Self-publishing through modern sales channels such as Google Play\footnote{\url{http://play.google.com}, last visited on 09/02/2013} or the iOS App Store \footnote{\url{http://appstore.com}, last visited on 09/02/2013} and the availability of easy-to-use Integrated Development Environments (IDEs) lower the barriers for entry. Even one-man developers or small teams are now able to publish apps with less development effort than a few years ago.\footnote{cf. for this and the next sentence \cite{Dehling.2013}, p. 2}\footnote{cf. for this and the next sentence \cite{Moore.2012}, p. 15} Without fundamental knowledge of privacy and security aspects, there is an increase in the non-professional developmental of mobile apps with possibly inadequate security aspects.
\\
\\
Usability as well as the overall app quality is also a possibly undervalued aspect in non-professional developments.\footnote{cf. \cite{Dehling.2013}, p. 2}\footnote{cf. \cite{Mayer.2012}, p. 1681} While fancy colors might look appealing to the developer himself, it might lead to confusion for the app user or even to a lack of operability for visually impaired people.\footnote{cf. \cite{Badashian.2008} p. 108} Also, the need for a intuitive user interface\footnote{for humans visible controls and layout of an application} has not been considered as important as it should be.
\\
\\
Knowledge of data privacy acts and laws is a premise for a legal, safe and fair development for the developer and the app user. Multiple layers of data privacy laws in Europe on international, national and state level require a certain legal knowledge.\footnote{cf. Directive 95/46 of the European Parliament and of the Council (October, 24th 1995), Directive 2002/58 of the European Parliament and of the Council (July, 12th 2002) cited by \cite{FutureofPrivacyForumCenterforDemocracy&Technology.2011}, p. 16} Also, the benefit of and the need for a privacy policy seems to be ambiguous for many non-professional developers.\footnote{cf. \cite{Njie.2013}, p. 20}
\\
\\
This lack of guidelines for mobile app development and of specific guidelines for privacy and usability sensitive apps is only superficially considered by most of the literature. The beforehand highlighted aspects of usability and information security\footnote{information security stands for prevention from unauthorized access, modification, use or disruption of information and information systems} are just two of multiple possible requirements. Current research seems not to state which specific requirements, if any, distinguish mHealth apps from other apps or which are needed to be more accented.

\subsection{Objectives of this Thesis}
The purpose of this thesis is to discover, identify and report issues and challenges of the development of mHealth apps by developing a mobile frontend for the ePill system\footnote{\url{http://epill.uni-koeln.de}, last visited on 09/24/2013} (developed by the University of Cologne). ePill is a patient-centered health IT service which offers information on pharmaceuticals and aggregation of pharmaceutical data in context.
\\
\\
During the development of a mobile frontend for ePill, all requirements can be addressed more easily than in a completely theoretical context. A mobile app for ePill will increase the accessibility for the ePill system in general, and thereby increase the possible user value. In critical situations in which one does not have a desktop computer at hand, a mobile easy-to-use app can be of value. It is important to us, that the developed app is at least accessible on Android and iOS devices.
\\
\\
The experiences gained during the development refer to general mobile app development, but also to the specific development of mHealth apps.
\\
Mainly this thesis aims to describe the planning and the development process and discuss all discovered issues and challenges for planning and developing mHealth apps. One sub-objective is to give a short overview about the state of research on guidelines and important factors of mHealth app development. We will validate the developed app by reference to the stated guidelines and best practices. Also we aim to provide an short overview over different frameworks, compare them and give a short evaluation about the framework we chose for the implementation of the mobile frontend for ePill.