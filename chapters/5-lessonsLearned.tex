\section{Lessons Learned}
\label{sec:LessonsLearned}
mHealth apps do have some similar requirements as any app, but have to deal with much stronger information security concerns compared to casual apps. They also need to pay more attention to usability and availability than most other apps by definition.\footnote{cf \cite{WorldHealthOrganization.2011} cited by \cite{MartinezPerez.2013}, p. 2}
\\
In the case of ePill information security is not an issue because no user related information is stored. But developing with Vaadin was not always a good choice. Vaadin simplified the development process but slowed it down at the very beginning. The app is not as perfectly designed as it could have been with more influence on the final layout on the client-side, especially on the user interface in terms of available user interface controls.
\\
\\
After having read the Vaadins Beginner Guide\footnote{\url{https://vaadin.com/tutorial}} we only had minor issues understanding Vaadins architecture and development went mostly quick and easy. As already stated, some missing controls were a drawback to us and workarounds had to be found, but all in all it is a good solution for a quick development without the need of further knowledge in HTML, CSS or JavaScript. Having not to deal with cross-browser-optimizations was a relieve.
\\
After the development of the mobile app we would nevertheless propose a different approach. If Vaadin is already utilized in the existing system, it is good to reuse the code. But if not, or if a web service is available, we would suggest developing native applications, maybe with Xamarin\footnote{\url{http://xamarin.com}}. This offers the possibility to fine tune the user interface much more, offer a much more familiar look on the different OS and much more user interface controls are available. While the web app developed with Vaadin looks on all OS similar and similar to iOS, with native apps the different apps would incorporate themselves more into the environment of the OS. It is worth the additional effort of developing a standalone web service to provide the data for a mobile client and different user interfaces, at least their OS-specific definition for the better accessibility. Additionally, OS like iOS offer native apps the possibility to run in background and perform specific tasks energy efficiently, while web-based apps are forced to quit on exit. This constraint does not affect ePill but e.g. monitoring apps need a continuing execution.
\\
Frameworks such as PhoneGap\footnote{\url{http://phonegap.com}} provide accessibility to extended functionality of the device for most mobile OS but still do not offer background execution.\todo{cite}
\\
Using native apps also eases the integration of accessibility features such as font enlargement or voice guided navigation inside the app. Most modern OS provide many accessibility features which can not or only partially be utilized inside a web-based app. A voice-guided navigation could be useful for ePill for either impaired people or in situations in which one is not able to operate by touch.
\\
\\
We furthermore learned that planning plays an important role for a fast and successful implementation. For a successful planning, knowledge of the used software and their possibilities is important. During planning we had only superficial knowledge about Vaadin and TouchKit, we inspected sample code and user interfaces, which made us believe that most controls from native applications can be utilized as well. If we knew during development that many controls are not implemented already, our planned user interface would have looked different and we would have saved some time during the implementation.