\section{The Development of the mobile Client}
\subsection{Preconditions}
\subsubsection{Norms for mobile Apps}
\footnote{\cite{BundesregierungderBundesrepublikDeutschland.01.03.2007}}\footnote{\cite{BundesregierungderBundesrepublikDeutschland.01.08.1996}}

\subsubsection{Best Practices}
\subsubsection{Internal requirements}
\subsection{Analysis}
\subsubsection{Assignment of a mHealth App Category}
The mobile app does not differ from the web application in terms of privacy risks, content or connectivity because it has exactly the same functions and does also not store any data and serves the same purpose for connectivity.
\\
We plan to implement every request to the server to be optimized for SSL-encryption as soon as the server is capable of accepting and responding with SSL-encryption. 
\\
\\
Therefor we would suggest to categorize the ePill mobile application as a low privacy risk, drug- and safety-related medical connectivity mHealth application.

\subsubsection{The different Operation Systems}
\paragraph{Android}
\paragraph{iOS}
\paragraph{Windows Phone 7 and 8}
\paragraph{other}
\subsubsection{Possible Frameworks and Technologies}
\paragraph{Xamarin}
\paragraph{Vaadin}
\paragraph{HTML 5, jQuery mobile and Phone Gap}
\paragraph{Completely native}
\subsubsection{The Choice for Framework XYZ}
\subsection{The Planning Process}
\subsection{(The Design Process)}
\subsection{The Implementation Process}
\subsection{Validation of the mobile Client}