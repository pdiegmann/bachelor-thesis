\section{The Development of the mobile Client}
\subsection{Preconditions}
\subsubsection{Norms for mobile Apps}
As already mentioned, ePill is currently only used in Germany, therefor we will focus on laws applicable in Germany. These laws are namely the \TKGns, the \TMGns, the \REG as well as the \DPA of \NRWns. The \TKG and \TMG are laws by state, whereas \REG is a european directive, specified by the respective Member States. 
\\
German federal states have their own \DPAsns. In this thesis we will focus on the \DPA of \NRW as ePill is located in \NRWns.
\\
\\
As the topmost layer of laws, the \REG defines more general directives. Article 4 defines national law applicable, if the natural or legal person, the controller\footnote{cf. \cite{TheEuropeanParliamentandtheCounciloftheEuropeanUnion.24.10.1995}, Article 2, (d)}, is located on a Member State's territory\footnote{cf. \cite{TheEuropeanParliamentandtheCounciloftheEuropeanUnion.24.10.1995}, Article 4, 1., (a) and (b)} or if any of the processing takes place on a Member State's territory\footnote{cf. \cite{TheEuropeanParliamentandtheCounciloftheEuropeanUnion.24.10.1995}, Article 4, 1., (c)}. Furthermore it is required, that the controller asks the user to consent to the use and collection of data\footnote{cf. \cite{TheEuropeanParliamentandtheCounciloftheEuropeanUnion.24.10.1995}, Article 7, (a)}, explicitly "data concerning health and sex life"\footnote{\cite{TheEuropeanParliamentandtheCounciloftheEuropeanUnion.24.10.1995}, Article 8, 1.} shall not be processed. Only if the user consents explicitly\footnote{cf. \cite{TheEuropeanParliamentandtheCounciloftheEuropeanUnion.24.10.1995}, Article 8, 2., (a)} or if the processing is done by a healthcare professional under national law and for preventive medicine, medical diagnosis or treatment or for the management of health-care services\footnote{cf. \cite{TheEuropeanParliamentandtheCounciloftheEuropeanUnion.24.10.1995}, Article 8, 3.}. 
\\
\\
This is refined by the the \TMGns. § 13, section (1) states, that the controller has to inform the user in a commonly understandable manner about the data which is collected and the form of processing of this data\footnote{cf. \cite{BundesregierungderBundesrepublikDeutschland.01.03.2007}, § 13, section (1)}. For a legal consent, the controller has to ensure, that the user is aware of his consent, that the consent is minuted, that the content of the consent is always available to the user and that the user can revoke his consent\footnote{cf. \cite{BundesregierungderBundesrepublikDeutschland.01.03.2007}, § 13, section (2)}.
\\
§§ 91, 93 and 94 of the \TKG states the same laws\footnote{cf. \cite{BundesregierungderBundesrepublikDeutschland.01.08.1996}, Section 2, §§ 91, 93, 94}.
\\
Also the \DPA of \NRW constitutes the same laws\footnote{cf. \cite{DerInnenministerdesLandesNordrheinWestfalen.09.06.2000}, Section 1, §§ 2, 4, 5}, with the only restrictions, that its scope is limited to \NRWns.
\\
\\
Therefor ePill should explicitly inform the user that no data is stored and only anonymized transacted to find matching results, to comply with the stated laws.

\subsubsection{Best Practices}
The World Wide Web Consortium (W3C) has published a document in 2008, which states the basic best practices for developing for the mobile web. This document states 60 best practices, which shall ensure a minimum quality level for mobile web applications. These best practices emphasize the need of regard of the device's capabilities and supported technologies\footnote{cf. \cite{WorldWideWebConsortium.2008}, e.g. 2., 11., 21., 42.}. 
\\
This document focuses on mobile web development\footnote{cf. \cite{WorldWideWebConsortium.2008}, Abstract}, which has of course differences to native app development (e.g. the usage of frames and the accessibility of the device's specific features), most of the best practices are applicable in both development environments.
\\
\\
For this specific project, which does not need more specific device capabilities, like positioning and navigation features, we can focus on best practices related to the user interface, input and navigation methods as well as general best practices. Depending on the framework chosen, some of the best practices are already dealt with by the framework or at least supported. E.g. a thematic consistency\footnote{cf. \cite{WorldWideWebConsortium.2008}, 1.} is provided by native apps by default and by frameworks such as the TouchKit for Vaadin as well. Although they can be overridden, they provide a consistent theme.
\\
Other best practices like utilizing a navigation bar at the page's top\footnote{cf. \cite{WorldWideWebConsortium.2008}, 8.} for the main navigation have already become a standard across different platforms and frameworks.
\\
\\
Best practices which are mainly determined by implementations of the developer, like the usage of colors\footnote{cf. \cite{WorldWideWebConsortium.2008}, 26., 27} or the chosen input methods\footnote{cf.\cite{WorldWideWebConsortium.2008}, 55., 56., 57.} are often supported by the different platforms or frameworks but cannot be guaranteed by those. Even if different input methods like a number pad for numeric inputs are provided by the framework or platform still need to be adapted and utilized by the developer to act in line with the best practices.
\\
\\
\cite{Wessels.2011}, \cite{Nicolaou.2013}, \cite{AyobNurulZakiahbinti.2009}, \cite{Dahanayake.2010}, \cite{Lica.2010}

\subsubsection{Internal requirements}
\subsection{Analysis}
\subsubsection{Assignment of a mHealth App Category}
The mobile app does not differ from the web application in terms of privacy risks, content or connectivity because it has exactly the same functions and does also not store any data and serves the same purpose for connectivity.
\\
We plan to implement every request to the server to be optimized for SSL-encryption as soon as the server is capable of accepting and responding with SSL-encryption. 
\\
\\
Therefor we would suggest to categorize the ePill mobile application as a low privacy risk, drug- and safety-related medical connectivity mHealth application.

\subsubsection{The different Operation Systems}
\paragraph{Android}
\paragraph{iOS}
\paragraph{Windows Phone 7 and 8}
\paragraph{other}
\subsubsection{Possible Frameworks and Technologies}
\paragraph{Xamarin}
\paragraph{Vaadin}
\paragraph{HTML 5, jQuery mobile and Phone Gap}
\paragraph{Completely native}
\subsubsection{The Choice for Framework XYZ}
\subsection{The Planning Process}
\subsection{(The Design Process)}
\subsection{The Implementation Process}
\subsection{Validation of the mobile Client}