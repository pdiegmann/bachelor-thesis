\addtocontents{toc}{\protect\newpage}    % add a line break to the table of contents

% 3. STATE OF THE ART ANALYSE
\section{State of the Art Analyse}
\label{sec:Analyse}

% 3.1 HERAUSFORDERUNGEN IM CLOUD COMPUTING (4 Seiten)
\subsection{Herausforderungen im Cloud Computing}
\label{sec:Herausforderungen}
Im Folgenden werden die in der Fachliteratur und den Konferenzbeiträgen seit Mai 2012 adressierten Herausforderungen im \TCCComma gruppiert nach ihren Themengebieten, erläutert.  

% 3.1.1 Business
\subsubsection{Business}
Im Themengebiet \emph{Business} wird als Herausforderung genannt, dass es sowohl für \CSU als auch für \CSP schwierig ist Geschäftsmodelle zu entwickeln\footnote{Vgl. \cite{Chard.2012}, S.551-563.}, alle auftretenden Kosten zu identifizieren\footnote{\cite{Son.2012}, S. 714.}\saveFN{\Son} 
und daraus ein Preismodell zu bilden um schlussendlich die Herausforderung einer Vertragsbildung\useFN{\Son} 
zu bewältigen. Oft fehlt es den \CSUn an Vertrauen\footnote{\cite{Garrison.2012}, S. 66-68.} in die vom \CSP angebotenen \CSs und ihre Servicequalität\footnote{\cite{Knapper.2012}, S. 3.}. 
Somit ist ein Risikomanagement\footnote{\cite{Martens.2012}, S. 871-872.} bei beiden Stakeholdern notwendig.
\newline
Der \CSP muss aber auch Lieferanten- und Kundenbeziehungen aufbauen und pflegen\footnote{\cite{Giessmann.2012}, S. 2, 16-17.}. Wohingegen der \CSU ein Bewusstsein für die Existenz der verfügbaren \CSs erlangen und eventuell bei der Entscheidung für ein \CC Angebot, durch Beratung, unterstützt werden muss.\footnote{\cite{Sim.2012}, S. 564-568.}
\newline
Für den \CSU ist es auch eine Herausforderung diese Adoptionsentscheidung\footnote{\cite{Garrison.2012}, S. 62.} zu fällen und die Furcht vor einem möglichen Kontrollverlust\footnote{\cite{Sundareswaran.2012}, S. 556-557.} über seine Daten als eine Barriere im Auslagern seiner Daten zu überwinden.

% 3.1.2 Zuverlässigkeit
\subsubsection{Zuverlässigkeit}
Die Herausforderungen im Themengebiet \emph{Zuverlässigkeit} liegen ausschließlich auf der Seite des \CSPsDot Er muss für größtmögliche Verfügbarkeit\footnote{\cite{He.2012}, S. 1345-1346.} und Fehlertoleranz\footnote{\cite{Zheng.2012}, S. 540-541.} seines \CSs sorgen, sodass \CSU sich auf sein Angebot verlassen können.
\newline
Durch die Allgegenwärtigkeit\footnote{\cite{Marston.2011}, S. 177.} eines \Cs s, also die Nutzung durch immer mehr internetfähige Geräte, sind hohe Anforderungen an die Leistung\footnote{Vgl. \cite{Liu.2012}, S. 2.} der \CC Infrastruktur und Software gesetzt, die der \CSP nur durch geeignete Virtualisierungstechnologien\footnote{\cite{Xiao.2012}, S. 1-2.} erfüllen kann.

% 3.1.3 Flexibilität
\subsubsection{Flexibilität}
Genau wie im Themengebiet \emph{Zuverlässigkeit} liegen auch die Herausforderungen im Themengebiet \emph{Flexibilität} im Aufgabenbereich des \CSPsDot
Ein Adoptionsargument für den \CSU ist die Anpassbarkeit\footnote{Vgl. \cite{Ackermann.2012}, S. 2.} des \Cs s an seine eigenen Bedürfnisse.
\newline
In das Themengebiet \emph{Flexibilität} fällt aber auch eine der zentralen Herausforderungen des \CCComma die \emph{Skalierbarkeit}\footnote{\cite{Xiao.2012}, S. 11-12.}. Der \CSU möchte seine allozierten Ressourcen möglicherweise in Abhängigkeit der Nutzungsnachfrage seines \CSs hoch- bzw. herunterskalieren um optimale Auslastungs- und Kostenstrategien nutzen zu können.

% 3.1.4 Interoperabilität
\subsubsection{Interoperabilität}
Für \CSU ist es interessant nicht nur an einen \CSP und seine Preispolitik gebunden zu sein.\footnote{\cite{Marston.2011}, S.182.} Aktuell existieren noch keine übergreifenden Cloud-Standards, die eine Interoperabilität sicherstellen könnten.\footnote{\cite{Villegas.2012}, S. 1331.} Eine weitere Herausforderung ist die Integration mehrerer \CSP in einem Verbund, sodass zu jedem Zeitpunkt der günstigsten Anbieter auswählt und genutzt werden kann.\footnote{\cite{Villegas.2012}, S. 1342.}

% 3.1.5 Privatsphäre und Nachgiebigkeit
\subsubsection{Privatsphäre}
Wie kann der \CSU in den \CS des \CSPs vertrauen gewinnen, ohne, dass er den \CSP persönlich kennt oder je kennenlernen wird? Dennoch soll der \CSU seine Daten auf eine fremde Infrastruktur, über eine unsichere Verbindung wie das Internet, auslagern und darauf vertrauen, dass diese vertraulich behandelt werden. Die Bildung von Vertrauen ist eine wichtige Herausforderung im Themengebiet \emph{Privatsphäre}. 
\newline
Durch regelmäßige Zertifizierung und Kontrolle der Sicherheitsstandards\footnote{Dieser und der folgende Satz folgen \cite{Wang.2012}, S. 220-222.} des \CSPs könnte eine Vertrauensbasis geschaffen werden. Eine unabhängige Institution zu finden, die diese Zertifikate ausstellt und kontrolliert ist eine weitere Herausforderung.

% 3.1.6 Ressourcenmanagement
\subsubsection{Ressourcenmanagement}
Eine der größten Herausforderungen für den \CSP ist das optimale Provisionieren von Ressourcen, das durch die schnell wechselnden Bedürfnisse der \CSU üblicherweise nur durch geeignete Virtualisierungstechnologien möglich ist.\footnote{\cite{Chaisiri.2012}, S. 164-165.} 
Der \CSP muss aber auch seine Netzwerk-Infrastruktur permanent überwachen um den richtigen Zeitpunkt der Provisionierung neuer Ressourcen abschätzen zu können.\footnote{\cite{Meng.2012}, S. 1-2.}
\newline
Auch für den \CSU stellt das Ressourcenmanagement eine Herausforderung dar. 
Der \CSU muss sich in \acs{IaaS} oder \acs{PaaS} Umgebungen um die optimale Ressourcenallokation kümmern und seinen \CS entsprechend der Auslastung optimal skalieren.\footnote{\cite{Papagianni.2013}, S. 1-2.} Bei einer möglichen Fehleinschätzung der benötigten Ressourcen kann es durch Überbelastung der Infrastruktur zu höheren Antwortzeiten oder Datenverlust und damit einer Minderung der Servicequalität kommen.\footnote{Vgl. \cite{Wang.2012}, S. 224-225.}

% 3.1.7 Sicherheit
\subsubsection{Sicherheit}
Für viele \CSU ist das Thema \emph{Sicherheit} und \emph{Vertraulichkeit} eines der ausschlaggebendsten Themen sich für oder gegen einen \CSP zu entscheiden.\footnote{Vgl. \cite{Wang.2012b}, S. 1.}
Der \CSP sieht sich also vor die Herausforderung gestellt eine geeignete Sicherheitsarchitektur\footnote{Vgl. \cite{Chard.2012}, S. 551-552.}, zum Beispiel  durch Verschlüsselung\footnote{\cite{Yu.2013}, S. 1-4.}, zu implementieren, sodass die Datenintegrität\footnote{\cite{Wang.2013}, S. 1-3.} und der Datenschutz\footnote{\cite{Chadwick.2012}, S. 1359-1361.} gewährleistet ist.
\newline
Es existiert ebenso die Möglichkeit, eine sichere Laufzeitumgebung für die virtuellen Maschinen des \CSUs im \CS des \CSPs zu schaffen.\footnote{Dieser und der folgende Satz folgen \cite{Li.2012}, S. 472-473.}
Unter der Annahme, dass die virtuellen Maschinen durch ein nicht vertrauenswürdiges Betriebssystem des \CSPs verwaltet werden, besteht die Herausforderung, eine, für den \CSUComma sichere Nutzung der virtuellen Maschinen dennoch möglich zu machen.

% 3.1.8 Verschiedenes
\subsubsection{Verschiedenes}
Durch die schnellen Ressourcenallokationen und das ebenso schnelle Freigeben von Ressourcen über Web-Schnittstellen ist der \CSP gezwungen auf ein Mindestmaß an Automatisierung zu setzen. Der Einflussfaktor "menschliche Interaktion" in diesen Ressourcenmanagement-Prozess wird also zunehmend zu einer Herausforderung und sollte minimiert werden, um niedrigere Reaktionszeiten erzielen zu können.\footnote{\cite{Kecskemeti.2013}, S. 1-2.}

% 3.2 ANALYSE UND EINORDNUNG DER HERAUSFORDERUNGEN (5 Seiten)
\subsection{Analyse und Einordnung der Herausforderungen}

% 3.2.1 Systematik
\subsubsection{Systematik}
\label{sec:Systematik}
Basierend auf den Leitlinien zum Anfertigen eines Literaturreviews von \cite{Webster.2002} wurde im Rahmen dieser Bachelorarbeit ein systematischer Review der Literatur aus dem \TCC von Mai 2012 bis April 2013 durchgeführt.\footnote{\cite{Webster.2002}, S. xiii-xxiii.}
Durchsucht wurden die Top 50 Journale des AIS Journal Rankings nach der Auflistung von \cite{Saunders.2005}.\footnote{\cite{Saunders.2005}.}
Ebenso wurden führende Informationssystem-Konferenzen im oben genannten Zeitrahmen durchsucht, wobei aufgrund der großen Ergebnismenge von einer detailreicheren Taxonomierung der Konferenzbeiträge im Rahmen dieser Bachelorarbeit abgesehen wurde. Dennoch wurde gezielt, aufgrund der geringen Artikelzahlen, nach Herausforderungen in der Integrations- und Außerdienststellungsphase in den Abstracts der Konferenzbeiträge gesucht, um diese, wie im Abschnitt \ref{sec:QuantAnalysis} erläutert, unteradressierten Lebenszyklusphasen aufzuwerten.
\newline
Durchsucht wurden die Online-Datenbanken EBSCOhost Academic Source Complete, IEEE Xplore, ProQuest, ACM Digital Library, ScienceDirect und AISEL. In den Feldern \emph{Titel}, \emph{Schlagwörter} und \emph{Abstract} wurde nach den Phrasen "Cloud", "Software as a Service", "Platform as a Service", "Infrastructure as a Service", "\acs{SaaS}", "\acs{PaaS}", "\acs{IaaS}" und "\acs{XaaS}" gesucht und die Ergebnisse in tabellarischer Form in Excel festgehalten.
Anschließend wurden die Ergebnisse gesichtet und auf ihre Relevanz geprüft. Ausgeschlossen wurden alle Artikel die keine konkreten Lösungsvorschläge zu \HiTCC enthielten oder durch ihre Spezialisierung nicht für einen Literaturreview geeignet waren. 
Um auch wirklich alle Artikel zu den \HiTCC zu finden wurde die Suche bewusst nur nach den Servicemodellen ausgeführt und dann später in der Excel Ergebnistabelle jeder der gefunden Artikel auf mögliche Adressierung von \HiTCC geprüft. Jeder Artikel wurde aufgrund seines Abstracts nach Lebenszyklus-Phase, Stakeholdern, Herausforderungen und Relevanz kategorisiert. Es wurden pro Artikel maximal drei Herausforderungen festgehalten, um nur die Kernherausforderungen der jeweiligen Artikel zu extrahieren.

% 3.2.2 Quantitative Analyse
\subsubsection{Quantitative Analyse}
\label{sec:QuantAnalysis}
Im Folgenden wird eine Auswertung der Ergebnisartikel, die für relevant befunden wurden, vorgenommen.
\newline
\begin{table}[h]
    \centering
    \begin{tabular}{|l|l|}
        \hline
        \textbf{Stakeholder} & \textbf{Anzahl} \\
        \hline
        \CSU & 29 \\
        \hline
        \CSP & 21 \\
        \hline
        \CSP und \CSU & 9 \\
        \hline
    \end{tabular}
    \caption{Stakeholder im \TCC}
    \label{tab:Stakeholder}
\end{table}
\newline
Tabelle 3-1 stellt die Anzahl der Artikel dar, die die jeweiligen Stakeholder-Gruppen adressieren. Die Verteilung ist mit 29 Artikeln, die Herausforderungen auf Seite des \CSU und 21 Artikeln, die \CSP Herausforderungen behandeln, relativ ausgewogen. Es bleibt kein Stakeholder signifikant unteradressiert. 9 Artikel beschreiben Herausforderungen die sowohl \CSP als auch \CSU betreffen.
\newline
\begin{table}[h]
    \centering
    \begin{tabular}{|l|l|}
        \hline
        \textbf{Lebenszyklus-Phase} & \textbf{Anzahl} \\
        \hline
        Akquise & 21 \\
        \hline
        Entwicklung & 13 \\
        \hline
        Integration & 8 \\
        \hline
        Operation & 16 \\
        \hline
        Außerdienststellung & 1 \\
        \hline
    \end{tabular}
    \caption{Lebenszyklusphasen von \CSs}
    \label{tab:Lebenszyklus}
\end{table}
\newline
Tabelle 3-2 zeigt die Anzahl der gefundenen Artikel, die den einzelnen Lebenszyklus-Phasen nach \cite{Schneider.2013} zugeordnet werden konnten. Hier zeigt sich eine deutlich unausgewogenere Verteilung. 
Während 21 Artikel sich mit der Akquise zwischen \CSUn und \CSPn beschäftigen und 13 Artikel die Entwicklung des \CC Angebots durch den \CSP beschreiben, wird die Phase der Integration des \CSs in die IT-Landschaft des \CSUs mit 8 Artikeln vergleichsweise gering adressiert. Die Operationsphase wird mit 16 Artikeln ähnlich oft wie die Akquisephase behandelt. Auffallend ist aber, dass die Außerdienststellungsphase in nur einem Artikel thematisiert wurde, beziehungsweise in keinem anderen Artikel mögliche Herausforderungen aus dieser Phase erwähnt wurden.
Diese Unteradressierung der Außerdienststellungsphase zeigt sich auch in den Ergebnissen, die von \cite{Esteves.2007} vorgestellt wurden.\footnote{Vgl. \cite{Esteves.2007}, S. 390.}
Auf mögliche fachliche Lücken in der Fachliteratur wird in Kapitel \ref{sec:Zukunft} näher eingegangen.
\newline
Die Herausforderungen, die in der wissenschaftlichen Fachliteratur durch das oben genannte Verfahren gefunden wurden, wurden bereits in Kapitel \ref{sec:Herausforderungen} aufgezeigt. Tabelle 3-3 listet die Anzahl der Artikel auf, die in den Literaturergebnissen gefunden wurden.
\newline
Besonders die Herausforderungen \emph{Kosten/Preissetzung}, \emph{Datenschutz} und \emph{Ressourcenmanagement} treten hervor, da diese Herausforderungen am häufigsten adressiert wurden.
Auffällig ist auch, dass aus dem Bereich \emph{Privatsphäre} die Herausforderung \emph{IT-Kontrolle / Zertifizierung} nur ein Artikel zu finden war, obwohl die Kontrolle von \CSs und deren Zertifizierung die Überwindung einer großen Barriere in der Nutzung von \CC wäre. \footnote{Vgl. \cite{Praeg.2006}, S. 8, 9.} 
\begin{table}
    \centering
    \begin{tabular}{|l|l|l|}
        \hline
        \textbf{Kategorie} & \textbf{Herausforderung} & \textbf{Anzahl} \\
        \hline
        \multirow{14}{*}{Business} & Adoptionsentscheidung & 5 \\ 
        \cline{2-3}
         & Bewusstsein & 3 \\ 
        \cline{2-3}
        & Geschäftsmodelle & 2 \\ 
        \cline{2-3}
        & Vertragsbildung & 1 \\ 
        \cline{2-3}
        & Kosten/Preissetzung & 12 \\ 
        \cline{2-3}
        & Entscheidungsunterstützung & 7 \\ 
        \cline{2-3}
        & Kontrollverlust & 1 \\ 
        \cline{2-3}
        & Verwaltbarkeit & 1 \\ 
        \cline{2-3}
        & Risikomanagement & 1 \\ 
        \cline{2-3}
        & Servicequalität & 2 \\ 
        \cline{2-3}
        & Vertrauen & 3 \\ 
        \cline{2-3}
        & Benutzeradoption & 1 \\ 
        \cline{2-3}
        & Lieferanten-/Kundenmanagement & 1 \\ 
        \hline
        \multirow{5}{*}{Zuverlässigkeit} & Verfügbarkeit & 4 \\ 
        \cline{2-3}
        & Fehlertoleranz & 2 \\ 
        \cline{2-3}
        & Leistung & 2 \\ 
        \cline{2-3}
        & Zuverlässigkeit & 4 \\ 
        \cline{2-3}
        & Virtualisierungstechnologie & 1 \\ 
        \hline
        Flexibilität & Skalierbarkeit & 3 \\ 
        \hline
        \multirow{3}{*}{Interoperabilität} & Integration & 1 \\ 
        \cline{2-3}
        & Interoperabilität & 2 \\ 
        \cline{2-3}
        & Standarisierung & 1 \\ 
        \hline
        \multirow{2}{*}{Verschiedenes} & Menschliche Interaktion & 1 \\ 
        \cline{2-3}
        & Gesamtframework & 1 \\ 
        \hline
        Privatsphäre & IT-Kontrolle / Zertifizierung & 1 \\ 
        \hline
        \multirow{8}{*}{Ressourcenmanagement} & Cloud-Betrieb & 3 \\ 
        \cline{2-3}
        & Datenqualität & 1 \\ 
        \cline{2-3}
        & Monitoring & 1 \\ 
        \cline{2-3}
        & Netzwerk und Infrastruktur & 2 \\ 
        \cline{2-3}
        & Ressourcenallokation & 5 \\ 
        \cline{2-3}
        & Ressourcenmanagement & 8 \\ 
        \cline{2-3}
        & Ressourcenprovisionierung & 6 \\ 
        \cline{2-3}
        & Virtualisierungstechnologie & 5 \\ 
        \hline
        \multirow{6}{*}{Sicherheit} & Vertraulichkeit & 2 \\ 
        \cline{2-3}
        & Datenintegrität & 2 \\ 
        \cline{2-3}
        & Datenschutz & 9 \\ 
        \cline{2-3}
        & Verschlüsselung & 2 \\ 
        \cline{2-3}
        & Sicherheit & 6 \\ 
        \cline{2-3}
        & Vertrauensunwürdiges Rechnen & 3 \\ 
        \hline
    \end{tabular}
    \caption{\HiTCC}
    \label{tab:Herausforderungen}
\end{table}
\newpage

% 3.3 LÖSUNGSVORSCHLÄGE (10 Seiten)
\subsection{Lösungsvorschläge}
\label{sec:Loesungsvorschlaege}
Im Folgenden werden die Lösungsvorschläge erläutert, die in den Ergebnisartikeln der Literaturrecherche beschrieben wurden. Die Lösungsvorschläge werden nach dem Lebenszyklusmodell von \cite{Schneider.2013}, wie in Kapitel \ref{sec:Lebenszyklus} bereits vorgestellt, gegliedert.

%
% 3.3.1 AKQUISEPHASE
%
\subsubsection{Akquisephase}
\label{sec:Lösung:Akquisephase}
Als eine wichtige Herausforderung in der Akquisephase zeichnet sich die \emph{Entscheidungsunterstützung} ab.\footnote{\cite{Schneider.2013}, S. 11.}
% ID 9, Oberle.2013
Um dieser Herausforderung entgegenzutreten schlagen \cite{Oberle.2013} eine \acf{USDL} vor, die eine exakte Beschreibung des durch den \CSP angebotenen \CS ermöglich.\footnote{Dieser und der folgende Satz folgen \cite{Oberle.2013}, S. 155-156.}
Dadurch kann der \CSU einen Automatismus entwickeln, der ihm, sofern sich eine genügend große Anzahl an \CSPn und ihre Cloud-Service-Spezifikationen durch eine \acs{USDL} ausdrücken lassen, bei der Entscheidung für oder gegen einen \CSP hilft.
Unter Cloud-Service-Spezifikationen verstehen \cite{Oberle.2013} eine maschinenlesbare Repräsentation von beispielsweise Preismodellen, Lizensierungsmöglichkeiten oder Verfügbarkeitsmerkmalen des \Cs s.
Kosten werden für den \CSU dadurch eingespart, dass er, durch die Automatisierung, mit verringerter menschlicher Interaktion, schneller die Angebote der \CSP vergleichen kann und den für ihn optimalen und preisgünstigsten \CSP finden kann.
\newline
% ID 26, Joshi.2012
Auch \cite{Joshi.2012} schlagen einen Automatismus für Cloud-Speicher-\linebreak Dienste vor, um aufgrund der vordefinierten Speicherkapazitätsbedürfnisse des \CSUs einen geeigneten \CSP zu finden und seinen \CS dann auch automatisch zu konsumieren.\footnote{\cite{Joshi.2012}, S. 10-13.}
\newline
% ID 19, Son.2012
Um diese Idee der Automatisierung noch zu verbessern entwickelten \cite{Son.2012} einen Verhandlungsmechanismus, der es möglich macht parallel mehrere, in Preis oder Nutzungszeitraum unterschiedliche, Vorschläge an den \CSP zu senden.\footnote{Dieser und der folgende Satz folgen\cite{Son.2012}, S. 713-719.}
Dies steht im Kontrast zu existierenden Mechanismen, in denen es immer nur möglich ist einen Preis- oder Zeitraumvorschlag einzeln zu senden.
Mit einem solchen Verhandlungsmechanismus ist es für den \CSU einfacher und effektiver eine Vertragsbildung einzugehen.\footnote{\cite{Son.2012}, S. 714.}
\newline
% ID 67, Martens.2012
Neben Kostenfaktoren nutzt das von \cite{Martens.2012} entwickelte mathematische Entscheidungsmodell auch den Faktor \emph{Risiko}, um sich für einen \CSP und dessen \CS zu entscheiden.\footnote{\cite{Martens.2012}, S. 871.}
Risiko modellieren \cite{Martens.2012} in Abhängigkeit des Vorhandenseins der Determinanten \emph{Integrität}, \emph{Vertraulichkeit} und \emph{Verfügbarkeit}.\footnote{\cite{Martens.2012}, S. 878, 883.}
\newline
% ID 114, Ackermann.2012
Um das Risiko eines \CC Projekts besser einschätzen zu können präsentieren \cite{Ackermann.2012} eine Zusammenfassung der möglichen Risiken im Themengebiet \emph{IT-Sicherheit}.\footnote{Vgl. \cite{Ackermann.2012}, S. 2-3.}
Risiko unterteilt sich nach \cite{Ackermann.2012} in sechs Dimensionen: Vertraulichkeit, Integrität, Verfügbarkeit, Performance, Verantwortlichkeit und Wartbarkeit.\footnote{\cite{Ackermann.2012}, S. 6.}
\cite{Ackermann.2012} schlagen vor den Risikowert einer \CC Adoptionsentscheidung anhand dieser Dimensionen zu bewerten.
\newline
\newline
% ID 103, Repschlaeger.2012
Die Anforderungen, die durch den \CSU an den \CS gestellt werden, sind abhängig vom Servicemodell, welches der \CS nutzt.\footnote{Dieser Abschnitt folgt \cite{Repschlaeger.2012}, S. 7-11.}
Dazu gehören die Zahlungsmöglichkeiten, das Preismodell, die Datensicherheit, Vertrauenswürdigkeit und Zuverlässigkeit des \Cs s, als auch die Flexibilität des Angebots und seine Rechenleistung.
Zusätzlich zu diesen Servicemodell-spezifischen Anforderungen kommen weitere Servicemodell-unabhängige Anforderungen hinzu. 
\cite{Repschlaeger.2012} berücksichtigen alle diese Anforderungen und entwickeln daraus ein Anforderungs-Framework, das "Cloud Requirement Framework", um für jedes Servicemodell die Adoption von \CSs zu erleichtern.
\newline
% ID 66, Mazhelis
Bei allen bereits genannten Methoden zur Entscheidungsunterstützung bezieht sich die Akquise der \CSs hauptsächlich auf Public Cloud Angebote, deren Umfang und Verfügbarkeit über das Internet geprüft werden kann. Dass Kostenoptimierungen auch für Hybrid Cloud Angebote sinnvoll sind zeigen \cite{Mazhelis.2012}, indem sie die Kosten von Rechenleistung und Netzwerkkommunikation betrachten und versuchen zu minimieren.\footnote{\cite{Mazhelis.2012}, S. 6.}
\cite{Mazhelis.2012} stellen fest, dass die Entscheidung für die Nutzung einer Hybrid Cloud, bestehend aus einer Zusammensetzung von Private Cloud und Public Cloud, bei konstanten Preisen pro allozierter \CC Einheit (Rechenleistungseinheit oder Speicherkapazitätseinheit) und des Netzwerkkommunikationsvolumens innerhalb des \Cs s, am kostengünstigsten ist.\footnote{Dieser und der folgende Satz folgen \cite{Mazhelis.2012}, S. 38.}
\newline
% ID 101, ElKihal.2012
Sinkt aber der Preis pro allozierter \CC Einheit, wenn mehr Einheiten genutzt werden, ist eine Hybrid Cloud nicht von Vorteil. Für den Fall einer Public Cloud schlagen \cite{ElKihal.2012} zwei Methoden vor, die die Abrechnungsverfahren von \acs{IaaS}-Angeboten transparenter darstellen sollen.\footnote{\cite{ElKihal.2012}, S. 1-2.}
\cite{ElKihal.2012} entwickeln zusätzlich zu der "hedonischen Preissetzungsmethode", die eine Beziehung zwischen Marktpreis des Produktes und seiner objektiv messbaren charakteristischen Eigenschaften, für jeden \CSP einzeln bewertet, eine weitere Methode.\footnote{Dieser und der folgende Satz folgen \cite{ElKihal.2012}, S. 4-6.}
Die "Preissetzungsplan-Vergleichs-Methode" nutzt zusätzlich zur "hedonischen Preissetzungsmethode" den Vergleich zwischen allen verfügbaren \CSPn und findet den preislich bestmöglichen \CSPDot
\newline
% ID 15, Sim.2012
Um aber einen Vergleich zwischen verschiedenen \CSPn herstellen zu können, muss zunächst ein Bewusstsein für die Anwesenheit von verschiedenen \CSs bestehen. \cite{Sim.2012} schlägt eine \CS Suchmaschine vor, die über Cloud Crawler\footnote{Ein Crawler, auch Spider genannt, ist ein Programm, dass das Internet nach neuen Einträgen für den Index einer Suchmaschine durchsucht, oder bereits bestehende Einträge erneuert. (\cite{Laudon.2010}, S. 386.)}
eine Datenbank mit Informationen über die gefundenen \CSs aufbauen und aktualisieren kann.\footnote{Dieser und der folgende Satz folgen \cite{Sim.2012}, S. 567-568.}
Mittels dieser Datenbank kann der von \cite{Sim.2012} entwickelte Verhandlungs-Agent das Ressourcenmanagement des \CSUs übernehmen und den optimal passenden \CSP aus dem Datenbestand der \CS Suchdatenbank auswählen.
\newline
% ID 31, Addis.2013
\cite{Addis.2013} gehen hier noch einen Schritt näher an die NIST Definition von \CC heran, die die höchste Verfügbarkeit bei minimaler menschlicher Interaktion als Kernelement von \CC definiert\footnote{Vgl. \cite{Mell.2011}, S. 2-3.}. \cite{Addis.2013} präsentieren ein Framework, dass die Ressourcenallokation für den \CSUComma während des laufenden Betriebs seiner Cloud-Anwendung, durchführt.\footnote{Dieser und der folgende Satz folgen \cite{Addis.2013}, S. 1-2.}
Die Erfolgsdeterminante, die \cite{Addis.2013} gesetzt haben ist, dass während das Ressourcenmanagement durchgeführt wird, der Betrieb der Cloud-Anwendung des \CSUs mit gleichen Verfügbarkeits- und Leistungsgarantien fortgesetzt und gleichzeitig die Minimierung der anfallenden Kosten angestrebt wird.
\newline
% ID 20, Chaisiri.2012 
Grundsätzlich gibt es zwei Arten von Ressourcenprovisionierung durch den \CSPDot Der \CSP kann dem \CSU die Reservierung von Ressourcen für einen bestimmten Zeitraum zu einem günstigeren Preis anbieten, als die Provisionierung der Ressourcen nach Bedarf ("on-demand") des Nutzers für kurze Nutzenzeiträume.\footnote{Dieser und der folgende Satz folgen \cite{Chaisiri.2012}, S. 164-165.} 
Das größte Problem beim Auswählen der optimalen Ressourcenprovisionierung ist, dass sowohl das zukünftige Nachfrageverhalten des \CSUsComma als auch die zukünftige Auslastung des \CSs nicht vorhergesagt, sondern nur approximiert werden können.
\cite{Chaisiri.2012} präsentieren einen Optimierungsalgorithmus, der sowohl eine längerfristige Ressourcen-Reservierung, als auch die kurzzeitige Nutzung der Ressourcen nach Bedarf in Betracht zieht, um die Gesamtkosten der Ressourcennutzung für den \CSU zu reduzieren.
\newline
% ID 106, Pueschel
Einen ähnlichen Ansatz verfolgen auch \cite{Pueschel.2012}, die eine realistische Abschätzung der Ressourcennachfrage sowohl mit klarer als auch mit unklarer Informationslage über die mögliche Auslastungsnachfrage durchführen.\footnote{Dieser und der folgende Satz folgen \cite{Pueschel.2012}, S. 3-4.} 
Sie legen eine dynamische Preissetzung für die Ressourcen fest und  erhöhen durch eine bessere Lastverteilung die Leistung des Systems.
\newline
% ID 48, Zhan.2012
Auch \cite{Zhan.2012} beschäftigen sich mit der Herausforderung des Ressourcenprovisionierens und schlagen ihr System "PhoenixCloud" vor, dass im Gegensatz zum Algorithmus von \cite{Chaisiri.2012} die Heterogenität der Arbeitsbelastung der Ressourcen mit einbezieht.\footnote{\cite{Zhan.2012}, S. 1-2} 
Unter heterogenen Arbeitsbelastungen verstehen \cite{Zhan.2012} parallele Stapelverarbeitungen, Web-Server, Suchmaschinen und MapReduce\footnote{MapReduce ist ein 2004 von Google eingeführtes System um große Datenmengen, über viele Server hinweg, effizient zu verarbeiten. (\cite{Marozzo.2012}, S. 1382.)} Aufgaben. 
Die verschiedenen heterogenen Arbeitsbelastungen nutzen aber auch verschieden intensiv bestimmte Ressourcencharakteristika, wie zum Beispiel Rechenleistung.\footnote{Vgl. für diesen und die folgenden zwei Sätze \cite{Zhan.2012}, S. 2.}
Auf Basis dieser Annahmen versucht das "PhoenixCloud" System möglichst effizient verschiedene Arbeitsbelastungen auf einer Server-Ressource ausführen zu können und optimiert so das Ressourcenmanagement. "PhoenixCloud" bezieht sowohl die Kosten für die Servernutzung, als auch die Energie- und Kühlinfrastrukturkosten in seine Berechnung mit ein.
\newline
% ID 17, Chard.2012
Ein interessantes Geschäftsmodell und gleichzeitig eine Ressourcenmanagement-Strategie bietet die Idee von \cite{Chard.2012}. 
Sie basiert auf der Annahme, dass Freundschaften in sozialen Netzwerken oft auf Freundschaften im wahren Leben beruhen.\footnote{Dieser Abschnitt folgt \cite{Chard.2012}, S. 551-552.}
Mit seinen wahren Freunden tauscht man gerne Informationen und Gegenstände aus, da man ihnen vertraut. \cite{Chard.2012} stellen ein Modell vor, dass sie "Social Cloud" nennen und sich diesen Freundschaftsgedanken zunutze macht. Ressourcen werden unter Freunden in der "Social Cloud" geteilt und gemeinsam genutzt. Damit adressieren \cite{Chard.2012} ebenso die Herausforderung der \emph{Vertrauensbildung}.
\newline
% ID 80, Garrison
Vertrauen wird auch im Artikel von \cite{Garrison.2012} als einer der wichtigsten Faktoren für den Erfolg von \CC Projekten beschrieben.\footnote{\cite{Garrison.2012}, S. 66.} 
Neben Managementfähigkeiten und technischem Fachwissen muss also ein Grundvertrauen zwischen \CSP und \CSU bestehen, damit der Cloud-\linebreak Service-Nutzer den \CS in seine IT-Landschaft aufnimmt. Diese drei Faktoren zusammen bilden in der Studie von \cite{Garrison.2012} 49\% der Varianz des \CC Projekterfolgs.\footnote{\cite{Garrison.2012}, S. 67-68.}
\newline
% ID 98, Giessmann.2012
Auch \cite{Giessmann.2012} haben sich mit Erfolgsfaktoren von \CC beschäftigt. Sie wollen Hinweise für Faktoren geben, die im Geschäftsmodell des \CSPsComma der \acs{PaaS}-Dienste anbietet, berücksichtigt werden sollten.\footnote{\cite{Giessmann.2012}, S. 2-3.}
\acs{PaaS}-Dienste sollten mindestens eine Verfügbarkeit von 95\% bieten, automatisch skalieren, eine standarisierte \acs{API}\footnote{Ein \acf{API} ist eine Schnittstelle über die auf die Funktionalität einer gekapselten Anwendung zugegriffen werden kann. Vor allem wird die Funktionalität von weiteren anderen Anwendungen konsumiert. (Vgl. \cite{Sommerville.2012}, S. 734.)}
zur Verfügung stellen, hohe Sicherheitsstandards und Zugriffskontrollen bieten und die Daten der Nutzer in sinnvollen Intervallen sichern, damit bei Bedarf diese Daten nicht verloren gehen, sondern wiederhergestellt werden können.\footnote{\cite{Giessmann.2012}, S. 15.}
Bietet der \CSP diese zwingend notwendigen Anforderungen an seinen \acs{PaaS}-Dienst, ist der Projekterfolg und eine Integration des \CSs durch \CSU sehr wahrscheinlich.
\newline
% ID 104, Bernius.2012
Einen Sonderfall von \CC  Adoptionsentscheidungen sehen \cite{Bernius.2012} in der Betrachtung von \acs{IaaS}-Umgebungen in Private Clouds oder Community Clouds für akademische Einrichtungen, einzelne Forscher oder kleine Forschungsgruppen.\footnote{\cite{Bernius.2012}, S. 2-3.}\saveFN{\Bernius} 
Sie sehen die Hauptmotivation in der Nutzung von Community Clouds oder Private Clouds für akademische Forschungsberechnungen in der Einfachheit der Nutzung und der Möglichkeit, schnell auf große Rechenleistungen zugreifen zu können, ohne eine, für die Forschungsgruppe oder den Forscher dedizierte Hardware über Tage in Anspruch zu nehmen.\footnote{Dieser und der folgende Satz folgen \cite{Bernius.2012}, S. 7.} 
Zudem ist die Nutzung von \CC Angeboten im akademischen Forschungskontext, nach \cite{Bernius.2012} eine relativ junge Idee, die neue Geschäftsmodelle hervorbringen könnte und in weiterer Forschung beleuchtet werden sollte.
\newline
% ID 115, Knapper.2012
Wie \cite{Giessmann.2012} sehen auch \cite{Knapper.2012} die Qualität des Angebots als entscheidenden Erfolgsfaktor in der Akquisephase. Um die Herausforderungen \emph{Zuverlässigkeit} und \emph{Servicequalität} zu adressieren, entwickeln \cite{Knapper.2012} ein Aggregations-Framework, dass es möglich macht die lose gekoppelte Struktur von Geschäftsprozessen und die einzelnen Subsysteme der Geschäftsprozesse zu aggregieren und ihre Servicequalität zu messen, um diese anschließend optimieren zu können.\footnote{\cite{Knapper.2012}, S. 3-4, 7.}
\newline
% ID 5, He.2012
Ein wichtiges Argument für den \CSU einen \CSP auszuwählen ist die Datensicherheit und Datenverfügbarkeit. \cite{He.2012} adressieren die Herausforderungen \emph{Datenintegrität} und \emph{Datenverfügbarkeit} mit ihrer Idee einer verteilten Datenbesitzprüfung.\footnote{Dieser und die nächsten beiden Sätze folgen \cite{He.2012}, S. 1345-1346.}
Ein oft genutzter Ansatz um weltweit-verteilte \CC Systeme zu betreiben, die die Latenzzeit\footnote{Latenzzeit ist die Verzögerung, die zwischen dem Zeitpunkt des Nachrichtenversands einer Komponente und dem Zeitpunkt des Empfangs der Nachricht bei einer anderen Komponente, anfällt. (\cite{Sommerville.2012}, S. 198.)}
möglichst gering halten sollen, ist Replikate des Datenbestandes geographisch verteilt zu speichern.
Durch den Ansatz von \cite{He.2012} wird es möglich die Datenintegrität und den Besitz der Daten über solche geographisch verteilten Clouds zu messen.
\newline
% ID 61, Suciu.2012
\label{id:61}Die vertrauensschaffende Komponente \emph{Datenschutz} hat auch \cite{Suciu.2012} aufgegriffen und schlägt vor die Daten des \CSUs verschlüsselt an den \CS zu übermitteln und dort zu speichern.\footnote{Dieser Abschnitt folgt \cite{Suciu.2012}, S. 102.}
Dies schützt sowohl vor ungewolltem Zugriff auf die sensiblen Daten durch den \CSPComma als auch vor Hackerangriffen auf den \CS und einem möglichen Datenzugriff durch diese Hacker.
Sind die Daten nun aber verschlüsselt im \CS gespeichert, kann zum Beispiel ein übliches Datenbanksystem keine Abfragen mehr bedienen. 
\cite{Suciu.2012} stellt seine Idee vor \acs{SQL}-Befehle\footnote{\acf{SQL} ist in relationalen Datenbanksystemen die standardmäßig verwendete Sprache um Daten abzufragen, einzufügen und zu modifizieren. (\cite{Laudon.2010}, S. 301.)} auf verschlüsselten Daten ausführbar zu machen und präsentiert dazu das Beispiel einer Tabelle mit dem Attribut \emph{Stadt}, wobei der Inhalt dieser Tabelle verschlüsselt ist. 
Anstatt nun alle Einträge der Spalte \emph{Stadt} zu entschlüsseln um eine beispielhafte SQL-Abfrage \emph{WHERE Stadt = 'Köln'} auszuführen, entwickelte \cite{Suciu.2012} die Idee, den Wert \emph{'Köln‘} mit dem gleichen Verfahren wie die in der Tabelle gespeicherten Daten zu verschlüsseln und den verschlüsselten Wert in der Where-Klausel (in diesem Beispiel) zu senden. 
Dadurch müssen keine entschlüsselten Daten auf dem \CS verarbeitet werden, was zur Gesamtsicherheit des Systems und der Abfragegeschwindigkeit beiträgt.
Zur Evaluierung der vorgestellten Idee verweist \cite{Suciu.2012} auf das Projekt "CryptDB"\footnote{\url{http://css.csail.mit.edu/cryptdb/}}, das genau diesen Ansatz benutzt, um Abfragen auf verschlüsselten Daten ohne temporäre Entschlüsselung auszuführen.
\newline
% ID 36, Yu.2013
Auch \cite{Yu.2013} adressieren die Herausforderung des Datenschutzes und sind der Überzeugung, dass die Daten eines \CSUs verschlüsselt im Cloud-\linebreak Service gespeichert werden sollten.\footnote{Dieser Abschnitt folgt \cite{Yu.2013}, S. 1-3.}
\cite{Yu.2013} präsentieren einen symmetrisches Verschlüsselungsverfahren\footnote{Bei einem symmetrischen Verschlüsselungsverfahren müssen beide Kommunikationspartner den gemeinsamen, geheimen Schlüssel besitzen, um die verschlüsselten Daten entschlüsseln zu können. (Vgl. \cite{Laudon.2010}, S. 1058-1059.)}, dass dem \CSU die Möglichkeit gibt eine Schlüsselwortsuche über die verschlüsselten Daten auszuführen. 
Vorteilhaft bei diesem Verschlüsselungsverfahren ist die Geschwindigkeit mit der auf die Daten zugegriffen werden kann. 
Durch das serverseitige Suchverfahren, über die verschlüsselten Daten, ist es möglich die üblichen Effizienzprobleme, die durch das Senden, Empfangen und Verarbeiten des kompletten verschlüsselten Datensatzes entstehen würden zu umgehen. 
Durch das Filtern, beziehungsweise Suchen auf verschlüsselten Daten, wird die Empfangs- und Verarbeitungszeit für den \CSU wesentlich reduziert.
\newline
% ID 18, Wang.2012 
Entgegen den Ansätzen der verschlüsselten Datenhaltung und Datenabfrage von \cite{Suciu.2012} und \cite{Yu.2013} stellen \cite{Wang.2012} ihren Ansatz die Daten unverschlüsselt auf dem \CS zu speichern, diese aber über einen Kontrollmechanismus zu überwachen.\footnote{Dieser und der folgende Satz folgen \cite{Wang.2012}, S. 220-221.}
Der Überwachungsmechanismus stellt mit geringer Rechnenleistung und Netzwerkbelastung Fehler in den gespeicherten Daten fest. Dadurch wird die Datenintegrität gesichert und sich nicht korrekt verhaltende Server identifiziert. Durch diese Prüfung lässt sich die Datenqualität kontinuierlich messen und gegebenenfalls korrigieren.
\newline
% ID 44, Li.2012
\cite{Li.2012} entwickeln eine Virtualisierungsarchitektur, die es möglich macht, virtuellen Maschinen, die sicherheitsrelevante Daten des \CSUs enthalten, eine Laufzeitumgebung in einer Public Cloud eines \CSPs zur Verfügung zu stellen, deren Management-Betriebssystem man nicht vertraut.\footnote{\cite{Li.2012}, S. 472-473.}\saveFN{\Li}
Das System enthält eine sichere Netzwerkschnittstelle, ein sicheres Festplattenspeichersystem und eine sichere Laufzeitumgebung für externe virtuelle Maschinen.\footnote{\cite{Li.2012}, S. 476.} Dazu haben \cite{Li.2012} das bereits existierende Virtualisierungssystem "Xen" so erweitert, dass es den Sicherheitsanforderungen einer vertrauenswürdigen Ausführungsumgebung genügt.\useFN{\Li}

%
% 3.3.2 ENTWICKLUNGSPHASE
%
\subsubsection{Entwicklungsphase}
\label{sec:LoesungEntwicklungsphase}
% ID 1, Wu.2012
Bei der Entwicklung von \acs{SaaS}-Diensten hat auch der \CSP die Wahl seine Infrastruktur selbst zu unterhalten oder auf zusätzliche \CC Infrastruktur von \acs{IaaS}-Providern zurückzugreifen um seine Kosten zu minimieren\footnote{Dieser und der folgende Satz folgen \cite{Wu.2012}, S. 1280-1281.}
Hier adressieren \cite{Wu.2012} sowohl die Herausforderungen \emph{Ressourcenmanagement} als auch speziell die Herausforderung der Ressourcenallokation von \CSPn unter Nutzung von Ressourcen weiterer \CSPDot
Um aber die unsichere Servicequalität der \acs{IaaS}-Provider so gering wie möglich zu halten, schlagen \cite{Wu.2012} einen Ablaufplanungsalgorithmus vor, der entscheidet, ob ankommende Anfragen an den \CS an die eigene oder die \CC Infrastruktur des externen \acs{IaaS}-Providers weitergeleitet werden. 
Die bereits hohen Qualitätsanforderungen an den \CSP werden an den \acs{IaaS}-Provider weitergereicht und sind dringendes Entscheidungskriterium für oder gegen einen externen \acs{IaaS}-Provider.\footnote{\cite{Wu.2012}, S. 1282.}
\newline
% ID 35, Papagianni.2013
Ebenso an die Herausforderung des Ressourcenmanagements von \acs{IaaS}-Providern gerichtet, präsentieren \cite{Papagianni.2013} einen Optimierungsalgorithmus, der das kosteneffiziente Bereitstellen von \CC Ressourcen, durch den \CSPComma sicherstellt.\footnote{Dieser und der folgende Satz folgen \cite{Papagianni.2013}, S. 1.} 
Das Optimierungsproblem bezieht die Qualitätsanforderung der \CSU mit ein, um ihnen genau die virtuellen Ressourcen zur Verfügung zu stellen, die sie zur Ausführung ihrer Anwendungen brauchen. Dazu bieten \cite{Papagianni.2013} ein Framework, dass als Parameter die Anfragen der \CSU nach virtuellen Ressourcen auf ein Optimierungsproblem der Ressourcenprovisionierung des \CSP anpasst.\footnote{\cite{Papagianni.2013}, S. 2.}
\newline
% ID 57, Kousiouris.2012
Auch \cite{Kousiouris.2012} beschäftigen sich mit der automatisierten Ressourcenprovisionierung und gehen noch einen Schritt weiter als der Optimierungsalgorithmus von \cite{Papagianni.2013}. 
\cite{Kousiouris.2012} schlagen ein künstliches neuronales Netz (\acs{KNN})\footnote{Ein künstliches neuronales Netz (\acs{KNN}) ist eine Technik des Maschinenlernens, die der Funktion des menschlichen Gehirns nachempfunden ist. Ein \acs{KNN} besteht aus verschiedenen Ebenen, die jeweils Knotenpunkte ("Neuronen") mit bestimmten Werten enthalten. Jedes Neuron ist über gewichtete Verbindungen mit einem oder mehreren Neuron aus einer anderen Ebene verbunden. Trainiert man das \acs{KNN}, bilden sich mit der Zeit stärkere und schwächere Verbindungen zwischen Neuronen heraus, denen man eine Semantik zuweist. (Vgl. \cite{Nallur.2012}, S. 19.)} vor, um die Nachfrage der \CSU über lange Sicht vorhersagen zu können.\footnote{\cite{Kousiouris.2012}, S. 1-2.}
Mit jeder Nachfrage durch einen \CSU und der anschließenden passenden Ressourcenprovisionierung des \CSPs kann das KNN trainiert werden und bei einer zukünftigen Nachfrage durch einen \CSU genauere Ergebnisse liefern.\footnote{\cite{Kousiouris.2012}, S. 4.}
\newline
\newline
% ID 13, Kecskemeti.2013
Virtuelle Maschinen zu verwalten ist eine übliche Funktionalität von \acs{IaaS}-Systemen.\footnote{Dieser Abschnitt folgt \cite{Kecskemeti.2013}, S. 1.}
Virtuelle Maschinen werden durch die Instanzierung und Kombination von virtuellen Abbildern von Softwarepaketen mit einem Betriebssystemen erstellt. 
So lässt sich zum Beispiel eine virtuelle Maschine aus dem Abbild einer Datenbanksoftware und dem passenden Betriebssystem erstellen. 
Als die zeitintensivste Komponente in diesem Vorgang identifizieren \cite{Kecskemeti.2013} die hocheffiziente Auslieferung dieser virtuellen Abbilder an den \CSU und schlagen vor, die virtuellen Abbilder auf minimal verwaltbare virtuelle Einheiten zu reduzieren, um diese schneller und effizienter auszuliefern, ohne eine Änderung des grundlegenden \acs{IaaS}-Systems vorzunehmen.
\cite{Kecskemeti.2013} präsentieren dafür einen Dienst, der automatisch virtuelle Abbilder aus einem Pool von minimal verwaltbaren virtuellen Abbildern zusammensetzt, um so schnell die vom \CSU gewünschte Anwendung zusammensetzen zu können.\footnote{Vgl. \cite{Kecskemeti.2013}, S. 7.}
\newline
% ID 14, Kangarlou.2012
Einen weiteren Beitrag zur Lösung einer Herausforderung aus den Themengebieten \emph{Virtualisierungstechnologie} und \emph{Zuverlässigkeit} leisten \cite{Kangarlou.2012} mit ihrer Idee zu Datensicherungszwecken Momentaufnahmen, sogenannte "Snapshots", der virtuellen Maschinen durch den \CSP speichern zu lassen\footnote{Dieser Absatz folgt \cite{Kangarlou.2012}, S. 484-485.}
Durch diese Momentaufnahmen lässt sich zu jeder Zeit der Stand der virtuellen Maschine zum Zeitpunkt der Aufnahme wiederherstellen. 
Die Besonderheit des Momentaufnahmesystems "VNSnap" von \cite{Kangarlou.2012} ist die Skalierbarkeit über eine verteilte \CC Infrastruktur und, dass an den virtuellen Maschinen, Anwendungen und Software-Bibliotheken keine Änderungen vorgenommen werden müssen, um "VNsnap" in Betrieb zu nehmen.\footnote{Vgl. \cite{Kangarlou.2012}, S. 494.}
\newline
\newline
% ID 43, Liu.2012 
Um den Qualitätsanforderungen von modernen Anwendungen wie Bank Transaktions Systemen oder der Datensynchronisation zwischen großen Kaufhausketten gerecht zu werden, müssen Daten mit Hochgeschwindigkeit transportiert werden.\footnote{Für diesen und den folgenden Satz vgl. \cite{Liu.2012}, S. 1.}
\cite{Liu.2012} identifizieren die statische Größe des Datenbuffers auf Empfängerseite als den limitierenden Geschwindigkeitsfaktor und präsentieren einen sich dynamisch anpassenden Empfangsbuffer "Rada" für den Hochgeschwindigkeitsdatenaustausch, um die Qualitätsanforderungen und Leistungsanforderungen der genannten modernen Anwendungen erfüllen zu können.
\newline
% ID 51, Klein.2013
Auch \cite{Klein.2013} adressieren die Latenzprobleme die bei höherer Netzwerknutzung durch \CC auftreten. Sie entwickeln ein Netzwerkmodell, dass durch einen Optimierungsalgorithmus die Latenzzeiten in der Komposition von \CSs minimiert und keine Einbußen in der Servicequalität macht.\footnote{\cite{Klein.2013}, S. 1, 3.}
\newline
% ID 93, Zhu.2012
\cite{Zhu.2012} widmen sich der Problematik fehlertoleranter und verlässlicher Kommunikation in \CC Umgebungen.
\cite{Zhu.2012} entwickeln einen Approximationsalgorithmus der ein Netzwerk aus optischen Glasfaserkabeln in Risikogruppen unterteilen kann, die Fehler verursachen könnten, und die Fehlerwahrscheinlichkeit über diese Knotenpunktgruppen minimieren kann.\footnote{Vgl. \cite{Zhu.2012}, S. 851.}
\newline
% ID 4, Marozzo.2012
Möchte man große Datenmengen über ein Servercluster\footnote{Ein Servercluster ist ein Verbund von Servern die geografisch so nah es geht verknüpft werden und für die Bewältigung von verteilten Rechenaufgaben genutzt werden. (Vgl. \cite{Laudon.2010}, S. 238.)} verteilt effizient verarbeiten, bedient man sich des von Google vorgestellten MapReduce Systems.\footnote{Dieser Abschnitt folgt \cite{Marozzo.2012}, S. 1382-1383.}
MapReduce basiert jedoch auf einer Master-Slave-Architektur\footnote{Eine Master-Slave-Architektur basiert auf der hierarchischen Idee, dass es einen primären Knotenpunkt gibt, der Aufgaben oder Ressourcen and sekundäre Knotenpunkte delegieren kann. (Vgl. \cite{Sommerville.2012}, S. 491.)}, die nur schwer in dynamischen und elastischen \CC Umgebungen einsetzbar ist, da die Existenz eines primären Master-Knotenpunktes gegen die Elastizität und Dynamik von \CC Infrastrukturen spricht.
\cite{Marozzo.2012} entwickelten daher ein Peer-to-Peer-System\footnote{In Peer-to-Peer-Systemen wird keinem Knotenpunkt (Peer) die alleinige Verantwortung für eine Aufgabe übergeben. Alle Peers können mit allen anderen Peers kommunizieren und sind in Rolle und Aufgabenverrichtung gleichberechtigt. (\cite{Laudon.2010}, S. 348.)} zur Ausführung von MapReduce Aufgaben.
Damit machen \cite{Marozzo.2012} das leistungsstarke MapReduce System in dynamischen und leistungsoptimierten \CC Umgebungen nutzbar und lösen die Herausforderung der Fehlertoleranz für den Spezialfall eines MapReduce-Systems.
\newline
% ID 55, Xu.2013
Damit aber fehlertolerante Anwendungen wie MapReduce Syteme verlässlich funktionieren können, müssen sie sich im Falle eines Systemausfalls auf Wiederherstellungsalgorithmen für Festplattenspeicher verlassen können.
\cite{Xu.2013} stellen "X-Code" vor, ein Fehlerwiederherstellungsschema, dass durch Datenredundanz die Verfügbarkeit der gespeicherten Daten sicherstellt.\footnote{\cite{Xu.2013}, S. 1-2.} 
Dieses Schema reduziert die Anzahl der Lesezugriffe auf die Festplatte, um Daten wiederherzustellen, um 25\% gegenüber bisher üblichen Verfahren.\footnote{\cite{Xu.2013}, S. 3.}
\newline
% ID 27, Wei.2012
Auch \cite{Wei.2012} adressieren die Herausforderungen der \emph{Skalierbarkeit} und \emph{Verfügbarkeit}. Sie stellen einen Transaktionsmanagager\footnote{Eine Transaktion ist eine Menge an Datenbankanweisungen, die gemeinsam auf einer konsistenten Datensicht der Datenbank ausgeführt werden. 
Scheitert eine atomare Anweisung der Transaktion werden alle anderen Anweisungen, zur Sicherung der Datenkonsistenz, nicht durchgeführt oder zurückgenommen. (Vgl. \cite{Wei.2012}, S. 1.)} für NoSQL-Datenbanken\footnote{NoSQL steht für "not only SQL" und beinhaltet laut \url{http://nosql-databases.org} alle Datenbanken, die nicht-relational, verteilt, quelloffen oder horizontal skalierbar sind.} in \CC Umgebungen, am Beispiel von \emph{Amazon SimpleDB}\footnote{\url{http://aws.amazon.com/simpledb}} und \emph{Google Bigtable}\footnote{\url{http://research.google.com/archive/bigtable.html}} vor.\footnote{Dieser Absatz folgt \cite{Wei.2012}, S. 1.}
Da NoSQL-Datenbanken üblicherweise eine geringe Datenkonsistenz aufgrund von Geschwindigkeit und Skalierbarkeit aufweisen, manche Anwendungen sich aber keine Dateninkonsistenz leisten können, sichert der Transaktionsmanager von \cite{Wei.2012} die benötigten Eigenschaften: Atomarität, Konsistenz, Isolation und Dauerhaftigkeit der Daten.
Des Weiteren adressieren \cite{Wei.2012} die Probleme in der Entwicklung eines dezentralen Transaktionsmanagers in \CC Umgebungen.
\newline
% ID 8, Chadwick.2012
In Kapitel \ref{sec:Lösung:Akquisephase} wurde \emph{Datensicherheit} als eine wichtige Herausforderung für eine Adoptionsentscheidung des \CSUs erwähnt. Eine Möglichkeit dem \CSU möglichst große Freiräume in der Definition seiner Sicherheitsrichtlinien zu bieten, stellt das Authorisierungssystem von \cite{Chadwick.2012} dar. Der \CSU definiert in diesem Authorisierungssystem seine Privatsphäre-Richtlinien und aufgrund deren, wie die Daten des \CSUs im \CS behandelt werden.\footnote{Dieser und der folgende Satz folgen \cite{Chadwick.2012}, S. 1359-1360.}
Durch diese individuelle Privatsphäre-Richtlinien Strategie ist der \CSU nicht mehr an die Datenschutz- und Privatsphäre-Einstellungen und -Richtlinien des \CSPs gebunden.
\newline
\newline
% ID 6, TolosanaCalasanz.2012
Üblicherweise können Workflows mit unterschiedlichen Qualitätsanforderungen nicht\linebreak 
gleichzeitig auf einer, von verschiedenen \CSUn genutzten \CC Infrastruktur ausgeführt werden, da die \CC Infrastruktur immer nur einer Untermenge aller Qualitätsanforderungen gerecht werden kann.\footnote{Vgl. zu diesem Abschnitt \cite{TolosanaCalasanz.2012}, S. 1300-1301.}
\linebreak
\cite{TolosanaCalasanz.2012} stellen eine Workflow-System-Architektur vor, die es möglich macht gleichzeitig mehrere Workflows, die jeweils an vordefinierte Qualitätsanforderungen gebunden sind, auf einer gemeinsam genutzten \CC Infrastruktur auszuführen. Dazu sichern \cite{TolosanaCalasanz.2012}, dass die Datenübertragung zwischen einzelnen Workflow-Instanzen den Qualitätsanforderungen der angehängten Workflows genügt.
\newline
% ID 86, Chang.2012a
Einen Sonderfall der Herausforderungen \emph{Skalierbarkeit} und \emph{Privatsphäre} stellt die von \cite{Chang.2012a} vorgestellte Architektur zur sicheren Übertragung von Videosignalen über das Internet, zum Betreiben von Video-Überwachungssystemen, dar.\footnote{Dieser und der folgende Satz folgen \cite{Chang.2012a}, S. 499-500.}
Das Besondere an diesem System ist, dass das Videosignal des \CSUs verschlüsselt wird, aber die Servicequalität und Geschwindigkeit der Videoübertragung unter dieser Sicherheitsmaßnahme nicht leidet.

% 3.3.3 Integrationsphase
\subsubsection{Integrationsphase}
% ID 30, Nallur.2012
In der Integrationsphase ist es für den \CSU wichtig, dass er den richtigen \CS in seine IT-Landschaft integriert. Die Idee von \cite{Nallur.2012} automatisiert diesen Prozess und erlaubt \CSUn über einen virtuellen Marktplatz den \CS auszuwählen, der den Qualitätsanforderungen des \CSUs gerecht wird.\footnote{\cite{Nallur.2012}, S. 72.}
\newline
% ID 107, Conboy.2012
Als weitere Herausforderung identifizieren \cite{Conboy.2012} das Training der Mitarbeiter des \CSUs in der Nutzung des neuen \CSs und erkennen, dass eines der Hindernisse bei der Integration von \CC der Mensch selbst ist. \footnote{Vgl. \cite{Conboy.2012}, S. 6.}
Als Empfehlung an \CSP nennen \cite{Conboy.2012}, den \CS so intuitiv und nutzerfreundlich wie möglich zu gestalten, damit die Adoption leicht fällt und eine höhere Akzeptanz des \CSs erreicht wird.\footnote{Dieser und der folgende Satz folgen \cite{Conboy.2012}, S. 7.}
Vor allem muss der \CSP aber ein Gefühl von Sicherheit bei den \CSUn vermitteln, sodass ein \CSU nicht das Gefühl hat die Kontrolle über seine Daten an den \CSP zu verlieren.
\newline
% ID 109, Retana.2012
Sollte es trotz guter Dokumentation und intuitiver Benutzerführung zu Problemen beim \CSU kommen bleibt nur die Möglichkeit den Support des \CSPs zu kontaktieren.
\cite{Retana.2012} haben über zweieinhalb Jahre hinweg die Nutzung des \CSP Supports durch 15.076 verschiedene \CSU beobachtet und ausgewertet.\footnote{Dieser und der folgende Satz folgen \cite{Retana.2012}, S. 2.}
Der \CSU konnte dabei zwischen zwei Support-Angeboten, "basic" und "managed" wählen, die sich nur durch den Hilfestellungsumfang durch das Support-Team des \CSPs unterschieden.
\cite{Retana.2012} beobachteten, dass jene \CSUComma die den umfangreicheren Support wählten im Durchschnitt 110\% mehr IT-Kapazitäten nutzten als die \CSU die lediglich das Basis-Support-Angebot auswählten.\footnote{\cite{Retana.2012}, S. 9.}
Außerdem ist den Beobachtungsdaten zu entnehmen, dass die \CSU mit erweitertem Support den \CS besser integrieren konnten und so größeren Nutzen aus ihm zogen.\footnote{\cite{Retana.2012}, S. 15.}
Der \CSP hat also höhere Chancen auf eine zufriedene Kundengruppe, die seinen \CS intensiv nutzt, wenn er einen tiefgreifenden Support anbietet.
\newline
\newline
% ID 89, Carey.2012
\cite{Carey.2012} beschreiben in ihrem Artikel das Thema \emph{Datendienste} und wie \CSU einen Mehrwert aus der Integration und Aggregation von externen Datendiensten, in Form von \acs{SaaS}-Diensten, generieren und diese wiederum bereitstellen können.\footnote{\cite{Carey.2012}, S. 1-2.}
Ein Datendienst ist die Bereitstellung von Unternehmensdaten über eine Software-Schnittstelle. Die Vorteile einen solchen Dienst in einer \CC Umgebung bereitzustellen sind die Abrechnung der anfallenden Kosten auf einer Nutzenbasis und die hohe Skalierbarkeit, sodass der Datendienst schnell von vielen \CSUn integriert und genutzt werden kann.\footnote{Vgl. \cite{Carey.2012}, S. 4-5.}
\newline
% ID 3, Qi.2012
Auch \cite{Qi.2012} sehen in der Komposition von Datendiensten in \CC Umgebungen, zu einem neuen \CSComma Herausforderungen im Themengebiet \emph{Servicequalität}.\footnote{Vgl. für diesen und den folgenden Satz \cite{Qi.2012}, S. 1316-1317.}
Da es eine große Menge an Qualitätsanforderungen der \CSU an den \CS gibt, der \CSP aber nur eine Untermenge dieser Qualitätsanforderungen in seinem \CS berücksichtigen kann, schlagen \cite{Qi.2012} eine Methode für den \CSU vor, um basierend auf den Qualitätsanforderungen des \CSUs die optimale Datendienst-Komposition zu finden. Nur durch diese Methode kann eine näherungsweise optimale Lösung für den \CSU gefunden werden.
\newline
% ID 2, Villegas.2012
Die Idee, wie auch bereits bei \cite{Wu.2012} in Abschnitt \ref{sec:LoesungEntwicklungsphase} beschrieben, als \CSP wiederum \CSs anderer \CSP zu nutzen und damit auch gleichzeitig zum \CSU zu werden, wird von \cite{Villegas.2012} aufgegriffen, um ein Modell für einen Verbund aus \CSs zu entwickeln.\footnote{Dieser und die folgenden zwei Sätze folgen \cite{Villegas.2012}, S. 1330.}
Das Cloud-Verbunds-Modell von \cite{Villegas.2012} setzt auf eine Integration von verschiedenen \CSs auf Ebene der Servicemodelle. Die einzelnen \CSs werden über einen Vermittlerdienst, pro Ebene, zu einem \CC Verbund zusammengeschlossen. 
Der Vorteil dieses integrierten Verbunds von verschiedenen \CSPn verschiedener Servicemodelle ist, dass sich jeder \CSP auf sein Spezialgebiet konzentrieren kann und, da er in diesem Gebiet über ein größeres Know-How verfügt, so einen besseren \CS liefern kann.\footnote{Vgl. \cite{Villegas.2012}, S. 1334.}
So muss zum Beispiel ein \CSP der einen \acs{SaaS}-Dienst anbietet sich nicht um die Verfügbarkeit seiner Infrastruktur kümmern, sondern nutzt für die Infrastruktur-Ebene einen \acs{IaaS}-Anbieter.
\newline
\newline
% ID 49, Zhou.2012
Einen interessanten Ansatz zur Lösung der Herausforderungen der \emph{menschlichen Interaktion} und \emph{Verwaltbarkeit} nutzen \cite{Zhou.2012} durch die Einführung eines asymmetrischen Überwachungssystem für virtuelle Maschinen.\footnote{Vgl. für diesen und den folgenden Satz \cite{Zhou.2012}, S. 2.}
Das Überwachungssystem teilt die zu überwachende virtuelle Plattform in zwei Teilpartitionen auf: eine Nutzerpartition und eine Servicepartition.
Die Nutzerpartition benutzt ein einfach zu bedienendes Betriebssystem mit grafischer Benutzeroberfläche, um es dem \CSU so einfach wie möglich zu machen die Ressource zu verwalten. Auf der Systempartition läuft ein spezielles Betriebssystem, das von der Ressourcennutzung komplett reduziert und leistungsoptimiert wurde.\footnote{\cite{Zhou.2012}, S. 3.}
Der Vorteil der einfacheren Handhabbarkeit der Ressourcen durch den \CSU und die optimierte Aufgabenerfüllung durch die Systempartition führen zu einer höheren Adoptionsrate und einer Reduzierung der Kosten, da die Mitarbeiter des \CSUs weniger im Umgang mit einem komplexen API oder kommandozeilenbasierten Oberfläche geschult werden müssen.\footnote{Vgl. \cite{Zhou.2012}, S. 1, 12-13.}
\newline
% ID 21, Tang.2012
Sollte der \CSP keine, für den \CSUComma ausreichenden Sicherheitsvorkehrungen für seinen \CC Speicherdienst anbieten, muss der \CSU eine Möglichkeit finden trotzdem seine Daten sicher im \CS des \CSP speichern zu können. 
Für diesen Fall stellen \cite{Tang.2012} das System "FADE" vor, das eine Zugriffskontrolle auf die in den \CS ausgelagerten Daten bietet und die sichere und vollständige Löschung der Daten auf dem \CS garantiert und überwacht.\footnote{Vgl. \cite{Tang.2012}, S. 903-904.}
Im "FADE"-System ist jede Datei direkt mit einer Dateizugriffs-Regel verbunden, über die feingranular die Datenschutzanforderungen für jede Datei gesichert werden kann.\footnote{\cite{Tang.2012}, S. 904-905.}
\cite{Tang.2012} haben das "FADE"-System testweise mit dem \CC Speicherdienst \emph{Amazon \acs{S3}}\footnote{\url{http://aws.amazon.com/de/s3}} evaluiert und erfolgreich getestet.\footnote{\cite{Tang.2012}, S. 910-911.} 

%
% 3.3.4 OPERATIONSPHASE
%
\subsubsection{Operationsphase}
% ID 24, Sundareswaran.2012
Auch in der Operationsphase kann der \CSP der Herausforderung der \emph{Datensicherheit} und der Wahrung der \emph{Privatsphäre} annehmen und so die Adoption seines \CSs für \CSU attraktiver gestalten.\footnote{Vgl. \cite{Sundareswaran.2012}, S. 556.}
Um dem \CSU mehr Transparenz über den Verbleib seiner Daten im \CS zu bieten, stellen \cite{Sundareswaran.2012} ein Framework vor, das dezentral die Nutzung der Daten im \CS in einer Logdatei aufzeichnet.\footnote{Dieser und der folgende Satz folgen \cite{Sundareswaran.2012}, S. 556-557.}
Das von \cite{Sundareswaran.2012} vorgestellte System stellt sicher, dass jedem Datenzugriff eine Authentifizierung des \CSUs vorausgeht und jeder Datenzugriff in eine entsprechende Logdatei geschrieben wird.
Dem \CSU bietet das System die Möglichkeit die Nutzung seiner Daten jeder Zeit kontrollieren zu können.
\newline
% ID 22, Wang.2013
Einen ähnlichen Ansatz wie \cite{Sundareswaran.2012} verfolgen \cite{Wang.2013} mit ihrer Idee ein \CC Speichersystem zu entwickeln, welches die Privatsphäre der Nutzerdaten des \CSUs durch einen Überwachungsmechanismus sichert.\footnote{Dieser Abschnitt folgt \cite{Wang.2013}, S. 1-2.}
Die Überwachung der Datenintegrität erfolgt durch einen unabhängigen dritten Prüfer, auf dessen Prüfung sich der \CSU verlässt.
Neben der Lösung der Herausforderung \emph{Datenintegrität} bieten \cite{Wang.2013} eine einfache und kostengünstige Lösung, die Privatsphäre der \CSU Daten zu kontrollieren und bieten dem \CSP eine Möglichkeit seine Vertrauenswürdigkeit offen zu kommunizieren.
\newline
% ID 7, Zhang.2012
Eine interessante Idee um die Privatsphäre eines \CSUs gegenüber \CSp n, die das Konsumverhalten ihrer \CSU ausspähen und analysieren, zu sichern, ist durch das konstante Senden von Störanfragen an den \Cs , die die eigentliche Intention des \CSUs verschleiern.\footnote{\cite{Zhang.2012}, S. 1-2.}\saveFN{\Zhang}
Bisherige Ansätze nutzen zufällig ausgewählte Störanfragen um das Verhalten des \CSUs zu verschleiern.\footnote{Vgl. für diesen und den folgenden Satz \cite{Zhang.2012}, S. 3.}
Dieser Ansatz macht sich jedoch in modernen \CC Bezahlmodellen mit Nutzungsabrechnungen deutlich bemerkbar, da auch jede Störanfrage einzeln bezahlt werden muss.
Daher stellen \cite{Zhang.2012} einen Algorithmus vor, der die historische Auftrittswahrscheinlichkeit einer Anfrage berechnet und so die minimale Anzahl an Störanfragen an den \CS sendet, um die Verteilung der verschiedenen Anfragen auszugleichen, was zu einer 90\% Reduzierung der Anfragen gegenüber zufälligen Störanfragen führt.\useFN{\Zhang}
Damit lösen \cite{Zhang.2012} sowohl die Herausforderungen \emph{Vertrauensschaffung} in \emph{unvertraulichen Rechenumgebungen}, als auch die Herausforderung des \emph{Datenschutzes.}
\newline
% ID 12, Chung.2013
Ein Sicherheitsproblem, das bisher nur auf Seite \pageref{id:61} von \cite{Suciu.2012} adressiert wurde, ist ein externer Hackerangriff. Da es für den \CSP sehr schwer ist seine \CC Infrastruktur permanent auf Eindringlinge oder Hackerangriffe zu untersuchen, stellen \cite{Chung.2013} einen Mechanismus unter der Namen "NICE" vor, der Schwachstellen des \CC Systems erkennt, misst und entsprechende Gegenmaßnahmen auswählt.\footnote{\cite{Chung.2013}, S. 1-2.}
Der "NICE" Mechanismus basiert auf einem Angriffs-Graphen, der in seiner Graphenstruktur mögliche Angriffsszenarien verkettet, die zu einem nicht wünschenswerten Zustand des \CC Systems führen können. Anhand dieser Graphenstruktur kann das "NICE" System von \cite{Chung.2013} die Gefahr einer Systemübernahme durch einen Angriff bewerten und effektiv überwachen.\footnote{Vgl. \cite{Chung.2013}, S. 2.}
\newline
\newline
% ID 54, Doulamis.2012
Auch in der Operationsphase steht der \CSP vor der Herausforderung eine optimale \emph{Ressourcenprovisionierung} zu bieten und der \CSU vor der Herausforderung die verfügbaren Ressourcen des \CSs optimal zu nutzen.
Einen bisher neuen Ansatz die Ressourcen des \CSPs bestimmten Aufgaben zuzuweisen bieten \cite{Doulamis.2012} an.\footnote{Dieser Abschnitt folgt \cite{Doulamis.2012}, S. 1-2.}
Eine Aufgabe definieren sie als Arbeitsprozess mit festgesetzter Start- und Endzeit. Der Start und das Ende der Aufgabe dürfen aber mit geringer Varianz von ihren ursprünglich festgesetzten Zeitpunkten abweichen.
Auf Basis dieser Annahmen entwickeln \cite{Doulamis.2012} einen Algorithmus, der diesen Aufgaben passende Ressourcen zuweist, sodass die genannten Zeiteinschränkungen möglichst selten verletzt werden und die Ressourcenanzahl so gering wie möglich bleibt.
Gleichzeitig versucht der Algorithmus die Auslastung der Ressourcen zu maximieren, um alle Aufgaben in kürzester Zeit abzuarbeiten. 
Mittels einer Graphenstruktur minimiert der Algorithmus die Überlappung der Startzeiten von Aufgaben, die sich eine Ressource teilen und maximiert die Überlappung der Startzeiten von Aufgaben die auf verschiedenen Ressourcen ausgeführt werden.
Damit werden die Gesamtdauer der Abarbeitung der Aufgaben durch optimale Parallelisierung verringert und effektiv Kosten eingespart.
\newline
% ID 77, GutierrezGarcia.2012
Einen ähnlichen Ansatz wie \cite{Doulamis.2012} liefern \cite{GutierrezGarcia.2012} durch einen generischen Algorithmus, der autonom Ressourcen zusammenstellen kann, die für die Ausführung von hochparallelisierbaren Aufgabengruppen, unter Berücksichtigung der Kosten und der Ausführungsfrist der Aufgaben, optimiert sind.\footnote{\cite{GutierrezGarcia.2012}, S. 925-926.}
\newline
% ID 53, Patel.2012
Der Herausforderung der \emph{Ressourcenallokation} nahmen sich auch \cite{Patel.2012} an und präsentieren ein Ressourcenallokations-Framework, das ein Profil der Energieanforderungen eines jeden neuen \CSUs erstellt und aus den Energieprofilen aller \CSU des \CSs einen Graphen erstellt.\footnote{Dieser und der folgende Satz folgen \cite{Patel.2012}, S. 1, 3.}
Das Framework findet anhand dieser Graphen-Struktur die optimale Ressourcen-Allkokationsstrategie bei Minimierung von Energiekosten und Maximierung des Profits.
\newline
% ID 79, Goiri.2012
Eine andere Lösungsstrategie der Herausforderungen aus dem Themengebiet \emph{Ressourcenmanagement} ist es,  als \CSP nur eine limitierte Menge an Ressourcen bereitzuhalten und in Zeitabschnitten, in denen mehr Ressourcen durch die Nutzung des \CSs durch \CSU benötigt werden, die zusätzlich benötigten Ressourcen von einem anderen \CSP hinzuzuziehen.\footnote{\cite{Goiri.2012}, S. 827-828.}
Durch diesen Zusammenschluss von \CSPn zu einem Verbund kann ein \CSP seine Ressourcenkosten für seine Stammkunden auf einem konstanten Niveau halten und nur bei Bedarf weitere Ressourcen anderer \CSP hinzumieten.
\newline
% ID 116, Hedwig.2012
Das System von \cite{Hedwig.2012} nutzt historische Nutzungsdaten um die zukünftige Auslastungen der \CC Infrastruktur des Cloud-Service-\linebreak Providers vorherzusagen, sodass der \CSP die approximativ optimale Ressourcenallokation vornehmen kann.\footnote{Dieser und der folgende Satz folgen \cite{Hedwig.2012}, S. 1-3. }
Der Vorhersagealgorithmus berücksichtigt bei seiner Bewertung der optimalen Ressourcen-Allokationsstrategie sowohl den Einfluss der Auslastungen auf die Leistung des gesamten \CC Systems, als auch den ökonomischen Einfluss der jeweils gewählten Strategien. 
\newline
% ID 102, Brandt.2012
Auch \cite{Brandt.2012} nutzen historische Nutzungsdaten um das Nutzerverhalten zukünftiger \CSUComma den Einfluss verschiedener Systemauslastungen auf die Systemleistung und den ökonomischen Einfluss verschiedener Ressourcen-\linebreak Provisionierungsstrategien vorherzusagen.\footnote{\cite{Brandt.2012}, S. 2.}
\cite{Brandt.2012} nutzen diese Vorhersagen, um \CSU entsprechend ihrer Qualitätsanforderungen in der DLV zu klassifizieren und die optimale Ressourcenallokation für diese \CSUComma auf Basis der Vorhersage, vorzunehmen.\footnote{Dieser und der folgende Satz folgen \cite{Brandt.2012}, S. 4.}
Somit ist das System in der Lage mehreren \CSUn mit verschiedenen \acs{DLV}s gleichzeitig gerecht zu werden.
\newline
% ID 37, Xiao.2012
Für den \CSU wiederum ist die Möglichkeit interessant, seine Anwendung über den \CS des \CSPs dynamisch skalieren zu können. 
\cite{Xiao.2012} modellieren das Problem der automatischen Skalierung von Ressourcen indem sie jede Anwendungsinstanz, die der \CSU in einer \CC Umgebung skalieren möchte, in einer virtuellen Maschine kapseln.\footnote{\cite{Xiao.2012}, S. 1-2.}
Um die optimale Skalierung zu finden wenden \cite{Xiao.2012} einen Optimierungsalgorithmus an, der ein klassenbeschränktes Behälterproblem\footnote{Das Behälterproblem ist ein Optimierungsproblem, dass kombinatorische gelöst werden kann. 
Es ermittelt die optimale Behälteranzahl, wobei ein Behälter feste Ausmaße hat, für eine Menge an  Objekten, mit variierenden Ausmaßen, sodass jedes Objekt in genau einem Behälter untergebracht wird. (Vgl. \cite{Voecking.2008}, S. 395-402.)} löst, wobei eine Klasse als eine Anwendung definiert ist und es ein physisches Limit für die Anzahl der Anwendungen pro Server-Ressource gibt.\footnote{\cite{Xiao.2012}, S. 1, 4.}
\newline
% ID 34, Zhao.2013
Exemplarisch an der Datenbankebene zeigen \cite{Zhao.2013} wie man ein skalierbares \CS Angebot in Abhängigkeit der \acs{DLV} zwischen \CSP und \CSU betreiben kann.\footnote{Vgl. für diesen Abschnitt \cite{Zhao.2013}, S. 1.}
Das von \cite{Zhao.2013} vorgestellte Framework überwacht den \CS des \CSPs permanent und führt bei möglichen Abweichungen von der \acs{DLV} Änderungen an der \CS Konfiguration durch. So ist es möglich, die Anforderungen des \CSUs auch während der laufenden Nutzung des \CSs zu erfüllen.
\newline
% ID 42, Mei.2012
\cite{Mei.2013} widmen sich der Leistungsoptimierung von, um Netzwerkbandbreite und Rechenleistung konkurrierenden, virtuellen Machienen pro physikalischer Server-\linebreak Ressource.\footnote{\cite{Mei.2013}, S. 1.}
Sie analysieren die Vorteile und Kosten von virtuellen Maschinen die außer Betrieb auf einer physikalischen Serverressource existieren, liefern eine detailreiche Analyse der Netzwerknutzung von konkurrierenden virtuellen Maschinen und untersuchen welchen Einfluss auf die Leistung verschiedene Verteilungstechniken von Prozessorleistung auf die virtuellen Maschinen haben.\footnote{Vgl. \cite{Mei.2013}, S. 1-2.}
\newline
\newline
% ID 33, Meng.2012
Um einen Überblick über den Status eines \CC Systems zu bekommen nutzen \CSU und \CSP Monitoring-Systeme. \CSP bieten \acf{MaaS} als \CC System an, um die Ressourcennutzung für das Monitoring skalierbar auf mehrere \CSU verteilen zu können.\footnote{Vgl. für diesen und den folgenden Satz \cite{Meng.2012}, S. 1.}
Klassisches Monitoring prüft üblicherweise periodisch den Status des zu überwachenden Systems und bietet dem \CSU die Auswertungen der Prüfungen als \acs{SaaS}-Dienst an.
\cite{Meng.2012} beschreiben einen effizienteren und kostengünstigeren Ansatz des Monitorings. So sollte das zu überwachende System nicht konstant periodisch überprüft werden, sondern beispielsweise nur in bestimmten festgelegten Zeitfenstern.\footnote{Dieser und der folgende Satz folgen \cite{Meng.2012}, S. 1-2.}
Denkbar wäre nach \cite{Meng.2012} auch die Wahrscheinlichkeit des Auftretens von wichtigen Ereignissen zu berechnen und nur in diesem berechneten Zeitfenster das zu überwachende System zu prüfen. 
\newline
% ID 16, Zheng.2012
Eine abgewandelte Art des Monitorings ist das Bewerten von einzelnen Komponenten in einem \CC System. Durch diese Methode bewerten \cite{Zheng.2012} alle Komponenten im \CC System und ermitteln die signifikantesten Komponenten.\footnote{\cite{Zheng.2012}, S. 540.}
Zur Lösung der Herausforderungen \emph{Fehlertoleranz} und \emph{Cloud Betrieb} schlägt die Methodik von \cite{Zheng.2012} Fehlertoleranz-Strategien für die signifikantesten Komponenten des \CC Systems vor, um die Stabilität des gesamten Systems zu verbessern.\footnote{\cite{Zheng.2012}, S. 541.}
\newline
\newline
% ID 52, Silic.2013
Oftmals setzen sich angebotene \Cs s, wie beispielsweise ein \acs{SaaS}-Dienst, aus einer Zusammenstellung von mehreren anderen \CSs zusammen.\footnote{Dieser und der folgende Satz folgen \cite{Silic.2013}, S. 1.}
In einem solchen Fall ist der \CSPComma der aus der Zusammenstellung von verschiedenen \CSs einen neuen \CS anbietet, abhängig von der Verfügbarkeit aller verbundenen \Cs s.
\cite{Silic.2013} präsentieren ein formales Modell "LUCS", das die Verfügbarkeit von anderen \CSs für den \CSP approximativ vorhersagen kann.\footnote{Vgl. für diesen und den folgenden Satz \cite{Silic.2013}, S. 2.}
Dazu nutzt "LUCS" die geographische Position des \CSUsComma die geographische Position des \Cs s, die Auslastung der \CS Infrastruktur und eine Klassifizierung der benötigten Arbeitslast der Anwendung. \cite{Silic.2013} zeigen, dass das "LUCS" Verfahren die Verfügbarkeit von \CSs durchschnittlich mit bis zu 81\% geringerer Fehlerrate vorhersagen kann, als herkömmliche Verfahren.\footnote{Vgl. \cite{Silic.2013}, S. 12.}
\newline
% ID 25, Zhu.2012
Um die Servicequalität einer \CC Anwendung zu erhöhen oder zu erhalten schlagen \cite{Zhu.2012} vor, eine Menge an Anpassungsparametern zu definieren, die den Wert einer anwendungsspezifischen Servicequalität-Funktion beeinflussen.\footnote{Dieser Absatz folgt \cite{Zhu.2012}. S. 497-498.}
Das von \cite{Zhu.2012} vorgestellte Framework kann, durch die Adjustierung der Anpassungsparameter, eine dynamische Anpassung der Servicequalität der Anwendungen in einer \CC Umgebung vornehmen. 
Kernpunkt des Frameworks ist ein zeitgenauer Ressourcen-Provisionierungsalgorithmus, der die Anpassungsparameter über einen Maschinenlern-Algorithmus, aufgrund von historischen Daten, adjustieren kann. 
Daraufhin wird die Ressourcenallokation aufgrund von Budget-Restriktionen optimiert, um am Ende die optimalen Ressourcen für die \CC Anwendung zur Verfügung stellen zu können.

%
% 3.3.5 AUßERDIENSTSTELLUNGSPHASE
%
\subsubsection{Außerdienststellungsphase}
\label{sec:LoesungenAusserdienststellungsphase}
\cite{Wang.2012b} hat sich mit der Datenkonsistenzprüfung und Löschung von Daten in Public Clouds beschäftigt, wenn der \CSU durch situative Einflüsse wie zum Beispiel einem Gefängnisaufenthalt, dem Aufenthalt auf einem Schiff auf hoher See oder durch den Ausbruch eines Krieges im Kriegseinsatz befindet und nicht in der Lage ist seine Daten selbst im \CS zu löschen.\footnote{\cite{Wang.2012b}, S. 2.}
Die Idee von \cite{Wang.2012b} ist es einen bevollmächtigten Dritten mit der Datenkonsistenzprüfung zu beauftragen, der sich um den Verbleib der Daten des \CSUs kümmert. Dafür präsentiert \cite{Wang.2012b} ein Protokoll, dass dem verhinderten \CSU die Sicherheit gibt, dass der bevollmächtigte Dritte tatsächlich in seinem Namen korrekt handelt.\footnote{Vgl. \cite{Wang.2012b}, S. 1.}

%
% 3.4 AUSBLICK AUF ZUKÜNFTIGE FORSCHUNG (3 Seiten)
%
\subsection{Ausblick auf zukünftige Forschung}
\label{sec:Zukunft}
Wie bereits in Tabelle 3-2 dargestellt, werden Herausforderungen aus der Integrationphase und der Außerdienststellungsphase in der wissenschaftlichen Fachliteratur von Mai 2012 bis April 2013 weniger oft adressiert, als Herausforderungen der anderen drei Lebenszyklusphasen nach \cite{Schneider.2013}.
\newline
Dieser Umstand bietet Potential für zukünftige Forschung, beispielsweise im Hinblick auf die Herausforderung \emph{Schulung von Mitarbeitern} des \CSUs im Umgang mit \Cs s, da dies zu einer höheren Akzeptanz des \CSs im Unternehmen des \CSUs führt.\footnote{Vgl. \cite{Conboy.2012}, S. 2.} 
\newline
In diesem Literaturreview wurde das Thema \emph{Schulung und Training} nur durch \cite{Conboy.2012} adressiert. 
\cite{Conboy.2012} thematisieren damit aber einen wesentlichen Punkt der Adoption von \Cs s. 
Abseits von technischen Herausforderungen wie der Stabilität der \CC Infrastruktur und der Sicherheitsarchitektur, die vergleichsweise häufig adressiert werden, wird in den gefundenen Artikeln das Themengebiet \emph{Mensch} beziehungsweise \emph{menschliche Interaktion} mit dem \CS vernachlässigt.
Der Mensch spielt aber eine entscheidende Rolle, da er meistens der direkte Nutzer des Systems ist. 
Direkt bedeutet, dass er zum Beispiel den in Abschnitt \ref{sec:SaaS} genannten E-Mail Dienst \emph{Gmail} über die Weboberfläche nutzen kann und soll. Der Mensch agiert hier, ohne einen zwischengeschalteten automatisierten Agenten, unmittelbar mit dem \Cs . 
Nun ist der E-Mail-Dienst \emph{Gmail} in der Benutzung kein allzu komplexes Produkt und verhältnismäßig leicht zu erlernen. Muss ein Mitarbeiter eines \CSUs aber einen Verbund aus verschiedenen \CSs verschiedener Servicemodelle Verwalten und Nutzen, kann es hilfreich sein vorher Trainingsprogramme der \CSP zu absolvieren. 
Wie diese Trainingsangebote aussehen könnten und optimalerweise aufgebaut sind oder ablaufen, sollte in zukünftiger Forschung berücksichtigt werden, denn \cite{Retana.2012} haben festgestellt, dass eine intensivere Kundenbetreuung durch den \CSP positiv mit einer intensiveren Nutzung des \CSs durch den \CSU korreliert.\footnote{Vgl. \cite{Retana.2012}, S. 15.}
\newline
\newline
Bis auf einen Artikel, wie in Tabelle 3-2 erläutert, wird die Außerdienststellung von \CSs in der wissenschaftlichen Fachliteratur von Mai 2012 bis April 2013 nicht behandelt. 
Dabei wäre es für \CSU eine wichtige Entscheidungsgrundlage, was mit ihren Daten nach der Nutzung eines \CSPs passiert. Möglicherweise entscheidet ein \CSU sich bewusst gegen einen \CSPComma wenn er weiß, dass seine Daten nach Ablauf des Nutzungszeitraumes nicht seinen Wünschen und Anforderungen entsprechend gelöscht werden.
\newline
\cite{Wang.2012b} hat sich mit der Problematik der Kontrolle des Datenbesitzes auseinandergesetzt, die entsteht, wenn der \CSU durch äußere Einflüsse und Gegebenheiten nicht mehr in der Lage ist, den Besitz seiner Daten zu prüfen und mit diesen Daten zu interagieren, also auch nicht in der Lage ist diese zu löschen.\footnote{Vgl. \cite{Wang.2012b}, S. 1.} 
Hiermit zeigt \cite{Wang.2012b} aber nur einen Sonderfall der Außerdienststellung.
\newline
Die Außerdienststellungsphase birgt also das größte Potential für zukünftige Forschung. 
Interessante Herausforderungen, die adressiert werden sollten sind für den \CSP die Frage nach passenden Strategien, wie dem \CSU die Entsorgung der Daten garantiert werden kann.\footnote{Dieser Abschnitt folgt \cite{Schneider.2013}, S. 17-18.}
Für den \CSU sollten Herausforderungen adressiert werden, die durch das Beenden des Geschäftsverhältnisses zu einem \CSP auftreten. 
Mögliche Alternativen für einen \CSU wären nach der Außerdienststellung eines \CSs die Nutzung der eigenen IT-Infrastruktur oder die Konsumierung eines  \CSs der von einem anderen \CSP angeboten wird.
Außerdem ist es für den \CSU eine Herausforderung, aus der Nutzung des außerdienstgestellten \CSs zu lernen, dieses Wissen zu konservieren und für zukünftige \CC Projekte zu nutzen.
Zum Beispiel könnte der \CSU die Faktoren oder Kriterien festhalten, die er als Erfolgsfaktoren in seinem \CC Projekt sieht. Daraus ergibt sich dann ein Maßstab für zukünftige \CC Projekte.
\newline
\newline
Eine phasenübergreifende Herausforderung ist die Zertifizierung und Kontrolle von \CSPn und ihren \Cs s, deren Bewältigung sich gleichzeitig positiv auf die Herausforderung des \emph{Vertrauens} in den \CSP auswirkt. 
In Tabelle 3-3 ist zu sehen, dass nur ein Artikel diese Herausforderung adressiert. 
Bezüglich der Zertifizierung von \CSs besteht ein Wissensdefizit in der Forschungsliteratur und dieses sollte durch zukünftige Forschungsarbeit adressiert werden.