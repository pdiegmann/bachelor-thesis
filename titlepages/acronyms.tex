\section*{Index of Abbreviations}
\addcontentsline{toc}{section}{Index of Abbreviations}
\begin{longtable}{@{}p{.275\textwidth}@{}p{.725\textwidth}@{}}
    API & Application Programming Interface. It specifies how software components could interact with each other. \\
    app & Application \\
    app user & intended audience for the app \\
    CDN & Content Delivery Network. Multiple servers which are globally distributed for serving static content with high availability and performance. \\
    CSS & Cascading Style Sheets. A language used to style web pages \\
    DNS & Domain Name System. Used to translate domain names into IP-Addresses \\
    eHealth & "a paradigm involving the concepts of health, technology, and commerce, with commerce and technology as tools in the service of health"\footnote{\cite{MartinezPerez.2013}, p. 2}. eHealth belongs to the field of telehealth.\footnote{cf. \cite{MartinezPerez.2013}, p. 2} \\
    ePill & a patient-centered health IT service which offers information on pharmaceuticals and aggregation of data in context\footnote{cf. \cite{Dehling.2012b}, p. 2} \\
    framework & can contain source code, tools and libraries, which together provide specific or common but abstracted functionality. \\
    frontend & visible user interface for the app user \\
    HECAT & Health Education Curriculum Analysis Tool\footnote{\url{http://www.cdc.gov/HealthyYouth/HECAT/}} \\
    HIT & Health Information Technology \\
    HTML & HyperText Markup Language, a markup language to design web pages. \\
    IDE & Integrated Development Environment \\
    JSON & JavaScript Object Notation, represents data structures \\
    mHealth & "medical and public health practice supported by mobile devices, such as mobile phones, patient monitoring devices, personal digital assistants (PDAs), and other wireless devices"\footnote{\cite{WorldHealthOrganization.2011} cited by \cite{MartinezPerez.2013}, p. 2}, also known as m-Health. \\
    mHealth apps & "aim at providing seamless, global access to tailored health IT services and have the potential to alleviate global health burdens"\footnote{\cite{Dehling.2013}, p. 1} \\
    MVC & Model-View-Controller. A software architecture pattern which separates logic and user interfaces. Models are representatives of data structures. Views contains the user interface definitions and controllers contains the application logic.\footnote{cf. \cite{Hasan2011} p. 418} \\
    NDK & Native Development Kit. Bundled software and tools which enables the developer to implement programs on native-code languages.\footnote{cf. \url{http://developer.android.com/tools/sdk/ndk/index.html}} \\
    OS & Operating System \\
    SDK & Software Development Kit. Bundled software and tools for developing with or for a specified OS or framework. \\
    telehealth & delivery of medical- or health-related information or services via telecommunication technologies. \\
    usability & "extent to which a product can be used by specified users to achieve specified goals with effectiveness, efficiency and satisfaction in a specified context of use"\footnote{\cite{Yeh.2012}, p. 64 as quoted from ISO 9241-11 (1998)} \\
    use value & the utility of consuming a good or service \\
    user interface & for humans visible controls and layout of an application \\
    W3C & World Wide Web Consortium\footnote{\url{http://www.w3.org}} \\
\end{longtable}