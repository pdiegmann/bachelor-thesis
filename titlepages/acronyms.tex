%Abkürzungsverzeichnis
\section*{Index of Abbreviations}
\addcontentsline{toc}{section}{Index of Abbreviations}
\begin{longtable}{@{}p{.275\textwidth}@{}p{.725\textwidth}@{}}
    app & abbreviation for Application\\
    app user & the intended audience for the app\\
    eHealth & "a paradigm involving the concepts of health, technology, and commerce, with commerce and technology as tools in the service of health"\footnote{\cite{MartinezPerez.2013}, p. 2}. Belonging to the field of telehealth.\footnote{cf. \cite{MartinezPerez.2013}, p. 2}\\
    ePill & a patient-centered health IT service which offers information on pharmaceuticals and aggregation of data in context\\
    framework & can contain source code, tools and libraries, which together provide specific or common but abstracted functionality\\
    frontend & visible user interface for the app user\\
    HIT & abbreviation for Health Information Technology\\
    IDE & abbreviation fro Integrated Development Environment\\
    mHealth & "medical and public health practice supported by mobile devices, such as mobile phones, patient monitoring devices, personal digital assistants (PDAs), and other wireless devices".\footnote{\cite{WorldHealthOrganization.2011} cited by \cite{MartinezPerez.2013}, p. 2} Also known as m-Health.\\
    mHealth apps & "aim at providing seamless, global access to tailored health IT services and have the potential to alleviate global health burdens." \footnote{\cite{Dehling.2013}, p. 1}\\
    information security & Prevention from unauthorized access to information. In this context especially sensitive, personal information\\
    OS & operating system\\
    SDK & abbreviation for software development kit. Bundled software and tools for developing with or for a specified OS or Framework\\
    sensitive information & information, which is personal. Can be related to financial, health or otherwise personal relevant information \footnote{Suggested by \cite{FutureofPrivacyForumCenterforDemocracy&Technology.2011}, p. 6, although the definition varies}\\
    telehealth & delivery of medical- or health-related information or services via telecommunication technologies\\
    usability & "extent to which a product can be used by specified users to achieve specified goals with effectiveness, efficiency and satisfaction in a specified context of use" \footnote{\cite{Yeh.2012}, p. 64 as quoted from ISO 9241-11 (1998)}\\
    use value & the utility of consuming a good or service\\
    user interface & \todo{Defintion!}\\
\end{longtable}